%%%%%%%%%%%%%%%%%%%%%%%%%%%%%%%%%%%%%%%%%
% Masters/Doctoral Thesis 
% LaTeX Template
% Version 2.5 (27/8/17)
%
% This template was downloaded from:
% http://www.LaTeXTemplates.com
%
% Version 2.x major modifications by:
% Vel (vel@latextemplates.com)
%
% This template is based on a template by:
% Steve Gunn (http://users.ecs.soton.ac.uk/srg/softwaretools/document/templates/)
% Sunil Patel (http://www.sunilpatel.co.uk/thesis-template/)
%
% Template license:
% CC BY-NC-SA 3.0 (http://creativecommons.org/licenses/by-nc-sa/3.0/)
%
%%%%%%%%%%%%%%%%%%%%%%%%%%%%%%%%%%%%%%%%%

%----------------------------------------------------------------------------------------
%	PACKAGES AND OTHER DOCUMENT CONFIGURATIONS
%----------------------------------------------------------------------------------------

\documentclass[
11pt, % The default document font size, options: 10pt, 11pt, 12pt
%oneside, % Two side (alternating margins) for binding by default, uncomment to switch to one side
english, % ngerman for German
singlespacing, % Single line spacing, alternatives: onehalfspacing or doublespacing
%draft, % Uncomment to enable draft mode (no pictures, no links, overfull hboxes indicated)
%nolistspacing, % If the document is onehalfspacing or doublespacing, uncomment this to set spacing in lists to single
%liststotoc, % Uncomment to add the list of figures/tables/etc to the table of contents
%toctotoc, % Uncomment to add the main table of contents to the table of contents
%parskip, % Uncomment to add space between paragraphs
%nohyperref, % Uncomment to not load the hyperref package
headsepline, % Uncomment to get a line under the header
%chapterinoneline, % Uncomment to place the chapter title next to the number on one line
%consistentlayout, % Uncomment to change the layout of the declaration, abstract and acknowledgements pages to match the default layout
]{MastersDoctoralThesis} % The class file specifying the document structure
\usepackage{xcolor,colortbl}
\usepackage[utf8]{inputenc} % Required for inputting international characters
\usepackage[T1]{fontenc} % Output font encoding for international characters

\usepackage{mathpazo} % Use the Palatino font by default

\usepackage[style=apa, backend=biber, natbib=true]{biblatex} % Use the bibtex backend with the authoryear citation style (which resembles APA)

\addbibresource{references.bib} % The filename of the bibliography

\usepackage[autostyle=true]{csquotes} % Required to generate language-dependent quotes in the bibliography

\usepackage[colorinlistoftodos]{todonotes}
\usepackage{amsmath}
\usepackage{enumitem}
\usepackage{caption}
\usepackage{subcaption}
\usepackage{tabularx,booktabs}
\usepackage{pifont}
\usepackage{float}
\newcommand{\cmark}{\ding{51}}
\newcommand{\xmark}{\textcolor{red}{\ding{55}}}
\newcolumntype{Y}{>{\centering\arraybackslash}X}
%----------------------------------------------------------------------------------------
%	MARGIN SETTINGS
%----------------------------------------------------------------------------------------

\geometry{
	paper=a4paper, % Change to letterpaper for US letter
	inner=2.5cm, % Inner margin
	outer=3.8cm, % Outer margin
	bindingoffset=.5cm, % Binding offset
	top=1.5cm, % Top margin
	bottom=1.5cm, % Bottom margin
	%showframe, % Uncomment to show how the type block is set on the page
}

%----------------------------------------------------------------------------------------
%	THESIS INFORMATION
%----------------------------------------------------------------------------------------
\thesistitle{Value-Sensitive Rejection of Machine Learning predictions for Hate Speech Detection} % Your thesis title, this is used in the title and abstract, print it elsewhere with \ttitle
\supervisor{Dr.\ J. Yang} % Your supervisor's name, this is used in the title page, print it elsewhere with \supname
\examiner{} % Your examiner's name, this is not currently used anywhere in the template, print it elsewhere with \examname
\degree{Master of Science} % Your degree name, this is used in the title page and abstract, print it elsewhere with \degreename
\author{Philippe Lammerts} % Your name, this is used in the title page and abstract, print it elsewhere with \authorname
\addresses{} % Your address, this is not currently used anywhere in the template, print it elsewhere with \addressname

\subject{Biological Sciences} % Your subject area, this is not currently used anywhere in the template, print it elsewhere with \subjectname
\keywords{} % Keywords for your thesis, this is not currently used anywhere in the template, print it elsewhere with \keywordnames
\university{\href{http://www.tudelft.nl}{Delft University of Technology}} % Your university's name and URL, this is used in the title page and abstract, print it elsewhere with \univname
\department{\href{https://www.tudelft.nl/en/eemcs/the-faculty/departments/software-technology/}{Software Technology}} % Your department's name and URL, this is used in the title page and abstract, print it elsewhere with \deptname
\group{\href{http://www.wis.ewi.tudelft.nl/}{Web Information Systems Group} - \href{https://www.wis.ewi.tudelft.nl/crowd-computing}{Crowd Computing}} % Your research group's name and URL, this is used in the title page, print it elsewhere with \groupname
\faculty{\href{https://www.tudelft.nl/en/ewi/}{Electrical Engineering, Mathematics and Computer Science}} % Your faculty's name and URL, this is used in the title page and abstract, print it elsewhere with \facname

\AtBeginDocument{
\hypersetup{pdftitle=\ttitle} % Set the PDF's title to your title
\hypersetup{pdfauthor=\authorname} % Set the PDF's author to your name
\hypersetup{pdfkeywords=\keywordnames} % Set the PDF's keywords to your keywords
}

\begin{document}

\frontmatter % Use roman page numbering style (i, ii, iii, iv...) for the pre-content pages

\pagestyle{plain} % Default to the plain heading style until the thesis style is called for the body content

%----------------------------------------------------------------------------------------
%	TITLE PAGE
%----------------------------------------------------------------------------------------

\begin{titlepage}
    \begin{center}

        \vspace*{.06\textheight}
        {\scshape\LARGE \univname\par}\vspace{1.5cm} % University name
        \textsc{\Large Masters Thesis}\\[0.5cm] % Thesis type

        \HRule \\[0.4cm] % Horizontal line
        {\huge \bfseries \ttitle\par}\vspace{0.4cm} % Thesis title
        \HRule \\[1.5cm] % Horizontal line

        \begin{minipage}[t]{0.4\textwidth}
            \begin{flushleft} \large
                \emph{Author:}\\
                \authorname % Author name - remove the \href bracket to remove the link
            \end{flushleft}
        \end{minipage}
        \begin{minipage}[t]{0.5\textwidth}
            \begin{flushright} \large
                \emph{Thesis advisor:}\\
                Prof. dr. ir. G.J.P.M. Houben\\
                Delft University of Technology
            \end{flushright}

            \begin{flushright} \large
                \emph{Daily supervisor:} \\
                Dr. ir. J. Yang\\
                Delft University of Technology
            \end{flushright}

            \begin{flushright} \large
                \emph{Co-daily supervisors:} \\
                Dr. ir. Y-C. Hsu\\
                University of Amsterdam\\
                \vspace{5mm}
                Ir. P. Lippmann\\
                Delft University of Technology\\
            \end{flushright}


        \end{minipage}\\[1.5cm]

        \vfill

        \large \textit{A thesis submitted in fulfillment of the requirements\\ for the degree of \degreename}\\[0.3cm] % University requirement text
        \textit{in the}\\[0.4cm]
        \groupname\\\deptname\\[1cm] % Research group name and department name

        \vfill

        {\large \today}\\[1cm] % Date
        %\includegraphics{Logo} % University/department logo - uncomment to place it

        \vfill
    \end{center}
\end{titlepage}

%----------------------------------------------------------------------------------------
%	DECLARATION PAGE
%----------------------------------------------------------------------------------------

% \begin{declaration}
% 	\addchaptertocentry{\authorshipname} % Add the declaration to the table of contents
% 	\noindent I, \authorname, declare that this thesis titled, \enquote{\ttitle} and the work presented in it are my own. I confirm that:

% 	\begin{itemize}
% 		\item This work was done wholly or mainly while in candidature for a research degree at this University.
% 		\item Where any part of this thesis has previously been submitted for a degree or any other qualification at this University or any other institution, this has been clearly stated.
% 		\item Where I have consulted the published work of others, this is always clearly attributed.
% 		\item Where I have quoted from the work of others, the source is always given. With the exception of such quotations, this thesis is entirely my own work.
% 		\item I have acknowledged all main sources of help.
% 		\item Where the thesis is based on work done by myself jointly with others, I have made clear exactly what was done by others and what I have contributed myself.\\
% 	\end{itemize}

% 	\noindent Signed:\\
% 	\rule[0.5em]{25em}{0.5pt} % This prints a line for the signature

% 	\noindent Date:\\
% 	\rule[0.5em]{25em}{0.5pt} % This prints a line to write the date
% \end{declaration}

% \cleardoublepage

%----------------------------------------------------------------------------------------
%	QUOTATION PAGE
%----------------------------------------------------------------------------------------

% \vspace*{0.2\textheight}

% \noindent\enquote{\itshape Thanks to my solid academic training, today I can write hundreds of words on virtually any topic without possessing a shred of information, which is how I got a good job in journalism.}\bigbreak

% \hfill Dave Barry

%----------------------------------------------------------------------------------------
%	ABSTRACT PAGE
%----------------------------------------------------------------------------------------

\begin{abstract}
    \addchaptertocentry{\abstractname} % Add the abstract to the table of contents
    Hate speech detection on social media platforms remains a challenging task. Manual moderation by humans is the most reliable but infeasible, and machine learning models for detecting hate speech are scalable but unreliable as they often perform poorly on unseen data. Therefore, human-AI collaborative systems, in which we combine the strengths of humans' reliability and the scalability of machine learning, offer great potential for detecting hate speech. While methods for task handover in human-AI collaboration exist that consider the costs of incorrect predictions, insufficient attention has been paid to estimating these costs. In this work, we propose a value-sensitive rejector that automatically rejects machine learning predictions when the prediction's confidence is too low by taking into account the users' perception regarding different types of machine learning predictions. We conducted a crowdsourced survey study with 160 participants to evaluate their perception of correct, incorrect and rejected predictions in the context of hate speech detection. We introduce magnitude estimation, an unbounded scale, as the preferred method for measuring user perception of machine predictions. The results show that we can use magnitude estimation reliably for measuring the users' perception. We integrate the user-perceived values into the value-sensitive rejector and apply the rejector to several state-of-the-art hate speech detection models. The results show that the value-sensitive rejector can help us to determine when to accept or reject predictions to achieve optimal model value. Furthermore, the results show that the best model can be different when optimizing model value compared to optimizing more widely used metrics, such as accuracy.
\end{abstract}

%----------------------------------------------------------------------------------------
%	ACKNOWLEDGEMENTS
%----------------------------------------------------------------------------------------

\begin{acknowledgements}
    \addchaptertocentry{\acknowledgementname} % Add the acknowledgements to the table of contents
    The acknowledgments and the people to thank go here, don't forget to include your project advisor\ldots
\end{acknowledgements}

%----------------------------------------------------------------------------------------
%	LIST OF CONTENTS/FIGURES/TABLES PAGES
%----------------------------------------------------------------------------------------

{\hypersetup{hidelinks}
\tableofcontents % Prints the main table of contents
}

% \listoffigures % Prints the list of figures

% \listoftables % Prints the list of tables

%----------------------------------------------------------------------------------------
%	ABBREVIATIONS
%----------------------------------------------------------------------------------------

\begin{abbreviations}{ll} % Include a list of abbreviations (a table of two columns)

    \textbf{BERT} & \textbf{b}idirectional \textbf{e}ncoder \textbf{r}epresentations from \textbf{t}ransformers\\
    \textbf{BOW} & \textbf{b}ag-\textbf{o}f-\textbf{w}ords \\
    \textbf{CNN} & \textbf{c}onvolutional \textbf{n}eural \textbf{n}etwork\\
    \textbf{DL} & \textbf{d}eep \textbf{l}earning\\
    \textbf{ECE} & \textbf{e}xpected \textbf{c}alibration \textbf{e}rror \\
    \textbf{FN} & \textbf{f}alse \textbf{n}egative\\
    \textbf{FP} & \textbf{f}alse \textbf{p}ositive\\
    \textbf{KDE} & \textbf{k}ernel \textbf{d}ensity \textbf{e}stimation \\
    \textbf{LDA} & \textbf{l}atent \textbf{d}irichlet \textbf{a}llocation \\
    \textbf{LR} & \textbf{l}ogistic \textbf{r}egression\\
    \textbf{LSTM} & \textbf{l}ong \textbf{s}hort-\textbf{t}erm \textbf{m}emory\\
    \textbf{ML} & \textbf{m}achine \textbf{l}earning\\
    \textbf{NLP} & \textbf{n}atural \textbf{l}anguage \textbf{p}rocessing\\
    \textbf{PDF} & \textbf{p}robability \textbf{d}ensity \textbf{f}unction \\
    \textbf{POS} & \textbf{p}art-\textbf{o}f-\textbf{s}peech \\
    \textbf{RR} & \textbf{r}ejection \textbf{r}ate\\
    \textbf{SVM} & \textbf{s}upport \textbf{v}ector \textbf{m}achine\\
    \textbf{TF-IDF} &  \textbf{t}erm \textbf{f}requency-\textbf{i}nverse \textbf{d}ocument \textbf{f}requency  \\
    \textbf{TN} & \textbf{t}rue \textbf{n}egative\\
    \textbf{TP} & \textbf{t}rue \textbf{p}ositive\\
    \textbf{VSD} & \textbf{v}alue-\textbf{s}ensitive \textbf{d}esign\\

\end{abbreviations}

%----------------------------------------------------------------------------------------
%	PHYSICAL CONSTANTS/OTHER DEFINITIONS
%----------------------------------------------------------------------------------------

% \begin{constants}{lr@{${}={}$}l} % The list of physical constants is a three column table

% 	% The \SI{}{} command is provided by the siunitx package, see its documentation for instructions on how to use it

% 	Speed of Light & $c_{0}$ & \SI{2.99792458e8}{\meter\per\second} (exact)\\
% 	%Constant Name & $Symbol$ & $Constant Value$ with units\\

% \end{constants}

%----------------------------------------------------------------------------------------
%	SYMBOLS
%----------------------------------------------------------------------------------------

% \begin{symbols}{lll} % Include a list of Symbols (a three column table)

%     % $a$ & distance & \si{\meter} \\
%     % $P$ & power & \si{\watt} (\si{\joule\per\second}) \\
%     %Symbol & Name & Unit \\

%     % \addlinespace % Gap to separate the Roman symbols from the Greek

%     % $\omega$ & angular frequency & \si{\radian} \\
%     \todo[inline]{Add symbols}

% \end{symbols}

%----------------------------------------------------------------------------------------
%	DEDICATION
%----------------------------------------------------------------------------------------

% \dedicatory{For/Dedicated to/To my\ldots}

%----------------------------------------------------------------------------------------
%	THESIS CONTENT - CHAPTERS
%----------------------------------------------------------------------------------------

\mainmatter % Begin numeric (1,2,3...) page numbering

\pagestyle{thesis} % Return the page headers back to the "thesis" style

% Include the chapters of the thesis as separate files from the Chapters folder
% Uncomment the lines as you write the chapters

\chapter{Introduction}
\label{ch:introduction}
\newcommand{\customtextbox}[1]{
	\setlength{\fboxsep}{0.5em}
	\fbox{
		\begin{minipage}{\linewidth-1.7em}
			\vspace*{0.25em}
			#1
		\end{minipage}
	}
}

The amount of hateful content spread online on social media platforms remains a significant problem. Ignoring its presence can harm people and even result in actual violence and other conflicts \citep{ecri-hate-speech-and-violence, balayn2021automatic}. There are many news articles about events where hate spread on online platforms lead to acts of violence \citep{columbia-facebook-linked-to-violence, mujib-mashal-india, paul-mozur-2018, muller2021fanning}. One research paper found a connection between hateful content on Facebook containing anti-refugee sentiment and hate crimes against refugees by analyzing social media usage in multiple municipalities in Germany \citep{muller2021fanning}. Governmental institutions and social media companies are becoming more aware of these risks and are trying to combat hate speech. For example, the European Union developed a Code of Conduct on countering illegal hate speech in cooperation with large social media companies such as Facebook and Twitter \citep{eu-code-of-conduct}. This Code of Conduct requests companies to prohibit hate speech and report their progress every year \citep{eu-code-of-conduct}. The most recent report from 2021 stated that Twitter only removed 49.5\% of all hateful content on their platform. Facebook is most successful in removing hate speech as they claim to have removed 70.2\% of all hateful content in 2021 \citep{eu-code-of-conduct}. However, one article found in internal communication from Facebook that this percentage is much lower, around 3-5\% \citep{noah2021giansiracusa}. Therefore, hate speech detection remains a hard problem that even large institutions have not solved yet.

Currently, people rely on reactive and proactive content moderation methods to detect hate speech \citep{klonick2017new}. Reactive moderation is when social media users are flagging (also known as reporting) hateful content \citep{klonick2017new}. Proactive moderation is either done automatically using detection algorithms or manually by a group of human moderators \citep{klonick2017new}. There exist different methods for automatically detecting hateful content. Most use Machine Learning (ML) algorithms since these tend to be the most promising for their detection performance at a large scale \citep{balayn2021automatic, fortuna2018survey}. These algorithms can range from traditional ML methods such as Support Vector Machine or Decision Tree to Deep Learning algorithms \citep{fortuna2018survey}.

However, both proactive and reactive moderation methods have their limitations. Proactive manual moderation of hateful content is still the most reliable solution but is simply infeasible due to the large amount of content generated by the many users \citep{balayn2021automatic}. Reactive moderation solves this problem since the users can report hate speech themselves. Although, the problem stays that hateful content is exposed to the users for some time. Proactive automatic moderation using automated detection algorithms allow for large amounts of data to be checked quickly without the involvement of humans. However, these algorithms have shown to be unreliable as they often perform poor on deployment data \citep{balayn2021automatic, grondahl2018all}. One study found that the F1 scores reduce significantly (69\% F1 score drop in the worst case) when training a hate speech detection model on one dataset and evaluating it using another dataset \citep{grondahl2018all}. Furthermore, one paper found that most research in hate speech detection overestimates the performance of the automated detection methods \citep{arango2019hate}. The authors found that the performance drops significantly when the detection algorithms are trained on one dataset and evaluated on another \citep{arango2019hate}.

This thesis research will tackle the problems of proactive moderation by focusing on the concept of \textit{human-machine co-creation} \citep{woo2020future} where the advantages of both humans (cognitive abilities and ability to make judgements) and machines (automation and performance) are combined. So humans and machines should work together to detect hate speech. ML models should detect hateful content automatically and humans should make the final decisions (\textit{human-in-the-loop}) when the model is not confident enough \citep{woo2020future}. Here come ML models with a reject option in place. The goal of the reject option is to reject an ML prediction when the risk of making an incorrect prediction is too high and to defer the prediction task to a human \citep{hendrickx2021machine}. There are several advantages. First, the utility of the ML model increases as only the most confident (and possibly the most correct) predictions are accepted. Second, less human effort is necessary as the machine is handling all prediction tasks, and only a fraction needs to be checked by a human. To the best of our knowledge, ML with rejection has not been used in hate speech detection before.

In this work, we focus on \textit{value-senstive} rejection. There are gains of accepting correct predictions (positive value) and costs of accepting incorrect or rejecting predictions (negative value). We should weigh these values according to the task of hate speech detection and incorporate them in the design of the hybrid human-AI system \citep{sayin2021science}. We will mainly focus on the user-centred value since the social media users are the most affected by the consequences of hate speech.

The idea of most ML models with rejection is that we reject predictions when the model's confidence is too low. Therefore, we need a metric that measures the total value of ML models with a reject option. We can use the resulting metric to determine when to reject/accept predictions by maximizing the total value. Second, we need to find out how we can define the user-centred values in the context of hate speech detection. We will attempt to retrieve the value ratios since it is hard to come up with the absolute cost values in the hate speech domain. By value ratios, we mean to figure out, for example, the ratio between an FP and an FN prediction. Therefore, our first sub-research question is as follows:

This leads to the following research questions:

% \noindent\customtextbox{\textbf{RQ} How can we maximize the value of Machine Learning models in hate speech detection using a reject option?}
% \noindent\customtextbox{\textbf{RQ} How can we reject predictions of Machine Learning models in hate speech detection in a value-sensitive manner?}
% \textbf{RQ} How can we maximize the value of Machine Learning models in hate speech detection using a reject option?

\noindent\customtextbox{
	\textbf{RQ} How can we reject predictions of Machine Learning models in a value-sensitive manner for hate speech detection ?
	\begin{itemize}
		\item \textbf{SRQ1} How can we measure the total value of Machine Learning models with a reject option?
		\item \textbf{SRQ2} How can we determine the value ratios between rejections and True Positive (TP), True Negative (TN), False Positive (FP), and False Negative (FN) predictions?
	\end{itemize}
}

% The second problem is about detecting the low and high confidence errors. These errors are also called \textit{unknown (un)knowns} \citep{liu2020towards}. When we would only rely on the confidence of the ML model to determine when to reject/accept predictions, then we would accept a subset of incorrect predictions with high confidence, and we would reject a subset of correct predictions with low confidence. We need to find a way to recognize these unknown (un)knowns. Once detected, we can reject the unknown unknowns so that a human moderator makes the final judgement and accept the unknown knowns to save the human moderator extra work. Doing so would further improve the utility of our smart rejector for detecting hate speech. So, our second sub research question is as follows:

% \todo[inline]{I will address the second sub research question only if there is enough time left.}

% \customtextbox{\textbf{SRQ2} How can we detect the unknown (un)knowns?}

% Finally, we need to find out how we can combine our findings into one smart rejection system which leads to our final sub research question:

% \customtextbox{\textbf{SRQ3} How can we build the smart rejector?}

\todo[inline]{Here comes a list of contributions}

\todo[inline]{Here comes a short description of the structure of the thesis report}

\chapter{Related work}
This section briefly defines hate speech and why it is such a challenging topic to tackle, especially in computer science.
%
Then, we give an overview of some algorithms we can use to detect hate speech automatically.
%
We provide examples of rejecting ML model predictions and discuss the main challenges of assessing the values of (in)correct and rejected predictions in the context of hate speech detection.
%
Finally, we discuss why standard metric, such as accuracy, are unsuitable to evaluate hate speech detection models and why we need human-centred metrics instead.

\section{Challenges of hate speech detection}
Different types of online conflictual languages exist, such as cyberbullying, offensive language, toxic language, or hate speech, and come with varying definitions from domains such as psychology, political science, or computer science \citep{balayn2021automatic}.
%
We can broadly define \textit{hate speech} as ``language that is used to express hatred towards a targeted group or is intended to be derogatory, to humiliate, or to insult the members of the group'' \citep{davidson2017automated, balayn2021automatic}.
%
It differs from other conflictual languages since it focuses on specific target groups or individuals \citep{balayn2021automatic}.
%
However, many more definitions exist in literature mainly because people differ on what is considered hate speech and what is not.
%
\citet{balayn2021automatic} identified the mismatch between the formalisation of hate speech and how people perceive it.
%
Many factors influence how people perceive hate speech, such as the content itself and the characteristics of the target group and the observing individual, such as gender, cultural background, or age \citep{balayn2021automatic}.
%
We can identify this mismatch in other related work from which there appears to be low agreement among humans regarding annotating hate speech \citep{fortuna2018survey, ross2017measuring, waseem2016you}.
%
\citet{ross2017measuring} found low inter-rater reliability scores (Krippendorff's alpha values of around $0.2-0.3$) in a study where they asked humans about the hatefulness and offensiveness of 20 tweets.
%
They also found that the inter-rater reliability value does not increase when showing a definition of hate speech to the human annotators beforehand.
%
\citet{waseem2016you} found a slight increase in the inter-rater reliability when considering annotations of human experts only, but it remained low overall.
%
% This low agreement makes sense since there are many differences in people's personalities and backgrounds.
%
Therefore, hate speech detection is challenging, especially in computer science, since we have to be careful with bias.
%
Most annotated hate speech datasets that are publicly available contain bias.
%
Annotating hate speech datasets is challenging because social media data follows a skewed distribution since there are many more neutral social media posts than hateful ones \citep{fortuna2018survey}.
%
Datasets such as \citet{waseem2016hateful} or \citet{basile2019semeval} collected their data using specific keywords that can introduce \textit{sample retrieval} bias and annotated their data using only three independent annotators that might result in \textit{sample annotation} bias \citep{balayn2021automatic}.
%
Automated classification algorithms will likely become biased in their predictions if we train them on biased datasets.
%
Bias becomes most notable when applying pre-trained classification algorithms to new and unseen data in deployment.
%
For example, \citet{grondahl2018all} and \citet{arango2019hate} report significant drops in F1 scores when training a hate speech detection model on one dataset and evaluating it on another.
%
\citet{grondahl2018all} found that the F1 score reduces by 69\% in the worst case and that the model choice does not affect the classification performance as much as the dataset choice.
%
\citet{arango2019hate} replicated several state-of-the-art hate speech detection models and found that most studies overestimate the classification performance.
%
These results further strengthen our stance that we should not detect hate speech solely by machines but rather by a human-in-the-loop approach.

\section{Detection algorithms}
There is an increasing academic interest in the automatic detection of hate speech since the topic has become more relevant, as explained in the \nameref{ch:introduction}.
%
We will list the state-of-the-art Natural Language Processing (NLP) techniques for automatic hate speech detection from literature.
%
First, we will list the different features used in the classification approaches.
%
Then, we will list the currently used classification algorithms ranging from supervised to unsupervised learning.
%

%
Several excellent surveys outlined the classification approaches from literature  \citep{fortuna2018survey, schmidt2019survey}.
%
Commonly used features are bag-of-words (BOW) \citep{greevy2004classifying}, character/word N-grams \citep{waseem2016hateful}, lexicon features (e.g. using a word blacklist containing offensive slurs) \citep{xiang2012detecting},  term frequency-inverse document frequency (TF-IDF) \citep{badjatiya2017deep, davidson2017automated, rodriguez2019automatic}, part-of-speech (POS) \citep{greevy2004classifying}, sentiment analysis \citep{rodriguez2019automatic}, topic modelling (e.g. Latent Dirichlet Allocation (LDA)) \citep{xiang2012detecting}, meta-information (e.g. location) \citep{waseem2016hateful}, or word embeddings \citep{badjatiya2017deep}.
%
\citet{greevy2004classifying} found that the classification performance is higher with BOW features than with POS features.
%
\citet{waseem2016hateful} found that character N-gram achieves higher classification performance than word N-gram.
%
They also found that using demographic information such as the location does not improve the results significantly.
%
\citet{xiang2012detecting} used the topic distributions from an LDA analysis and a lexicon feature.
%
\citet{rodriguez2019automatic} used TF-IDF and sentiment analysis to detect and cluster topics on Facebook pages that are likely to promote hate speech.
%
\citet{badjatiya2017deep} experimented with different word embeddings: fastText\footnote{\url{https://fasttext.cc/}}, GloVe\footnote{\url{https://nlp.stanford.edu/projects/glove/}}, and random word embeddings.
%

Most hate speech-related studies use supervised learning techniques that range from traditional ML to deep learning classification models, and a few use unsupervised learning techniques to cluster social media posts.
%
Support Vector Machine (SVM) \citep{greevy2004classifying, xiang2012detecting,davidson2017automated} and Logistic Regression (LR) \citep{waseem2016hateful, davidson2017automated} are the most popular traditional ML techniques for hate speech detection.
%
\citet{davidson2017automated} found that SVM and LR perform significantly better than traditional ML techniques such as Naive Bayes, Decision Trees, and Random Forests.
%
\citet{badjatiya2017deep} experimented with various configurations of word embeddings and two deep learning models: a convolutional neural network (CNN) and a long short-term memory (LSTM) model.
%
They found that CNN performs better than LSTM and that using pre-trained word embeddings such as GloVe does not result in better classification performance than using random embeddings.
%
\citet{rodriguez2019automatic} use the unsupervised learning method, K-means clustering, to cluster social media posts to identify topics that potentially promote hate speech.
%


\section{Machine Learning models with rejection}
\todo[inline]{Explain the different architectures of ML with rejection}
\todo[inline]{Explain the different types of confidence metrics}
\todo[inline]{Explain the original metric from De Stefano}
\todo[inline]{Provide examples of ML models with rejection from other domains}

\section{Value assessment}
\todo[inline]{Explain what we mean by value and why we should integrate it into the design of a hate speech detection system.}


\todo[inline]{The authors of \citet{fjeld2020principled} outlined 8 principles of AI systems including fairness and discrimination (e.g. algorithmic bias), human control of technology (e.g. AI system should request help from the human user in difficult situations), and promotion of human values (we should integrate human value in the AI system).}

\todo[inline]{We need to weigh the value of (in)correct and rejected predictions into account in the design of a hybrid human-AI system \citep{sayin2021science}}

\todo[inline]{Value sensitive Design (VSD) from \citet{umbrello2021mapping} state that we can translate values such as freedom of bias into the design of a system. So example is a tax system that needs to detect fraud. If etniticity bias can be introduced by using postal codes, than we can exclude the postal code variable from the learning algorithm. We for example have the value reliability. The AI is not reliable sometimes, so we use reject option in our design to make the use of AI more reliable/. The authors say that we not only want to optimize the tax algotihm in terms of effectiveness (rate of fraud detection) but also in terms of fairness (presenting non-biased selection of cases. So the same holds in our case, we want to optimize in using the AI as much as possible but also want to take mistakes into account (in the end we still need to invovle humans to check specific cases).}

\todo[inline]{The Value-design algorithm paper from \citet{zhu2018value} describes 5 steps of algorithm design. We focus our approach on theirs by inspecting the stakeholders. And then assessing the stakeholder values to take this into account in the algoithm design.}



In this research, we focus on Machine Learning models with a reject option. The decision to accept or reject predictions depends heavily on the context. We argued in the \nameref{ch:introduction} that this decision should depend on the costs of incorrect predictions and the gains of correct predictions. We can express the costs of incorrect (FP and FN predictions) and rejected predictions as negative values. The gains of correct predictions (TP and TN predictions) as positive values. In some domains, we can define these values in money or time. For example, suppose there is a factory that uses a camera and an ML model to check if a package is damaged or not. Using an ML model will save the company time since these packages no longer have to be inspected manually by humans. However, the ML model could be incorrect sometimes. For example, a customer of the factory received a damaged package, while the ML model did not detect any damage. Fixing this issue could cost the factory money. At the same time, the factory could prevent these cases by rejecting the low confidence predictions from the ML model. For example, the ML model predicted with low confidence that a package did not contain any damage. An employee can then inspect it to prevent the customer from receiving a damaged one. Manually checking the rejected ones costs the factory a fraction of the time/money compared to the first situation. In this example, we can easily express the values of FP, FN, TP, TN, and rejections in time/money spent/saved.

However, it is not evident to express these values in the hate speech domain. Two stakeholders can be considered in the design of a smart rejector: the social media company and its users. We mainly focus on the users in this research since they will be affected the most by hate speech.

In this section, we will look at the related work to get an understanding of how we could retrieve the value ratios in hate speech detection. The goal is to retrieve ratios between rejection, FP, FN, TP, and TN cases. We would like to know whether an FN is, for example, two times worse compared to an FP. The main challenge is to express all values using a single unit. We could take two directions. First, we could define the values using an objective measure, such as time or money spent/saved. Second, we could define the values subjectively, e.g. by analyzing people's stance towards the consequence of incorrect predictions in hate speech detection. In the next two sections, we discuss the relevant related work in both directions.

\subsection{Objective assessment}
In this section, we explain the difficulties of defining the values using objective measurements. We do this by looking at some related work. We can retrieve the value of rejection by looking at how much time a human moderator spends on average to check whether some social media post contains hateful content or not. We can convert this into money by taking the moderator's salary into account. We could also argue that the value of a TP and a TN is equal to the negative value of rejection since we saved human effort by letting the ML model correctly predict whether something is hateful or not. The problem, however, starts to arise when we look at the FP and the FN predictions. How can we express the values of FP and FN predictions in terms of money or time?

First, we look at the social media company as a stakeholder. As we explained in the previous section, the values of rejection, TP, and TN can be determined. So the values of FP and FN are yet to be defined. However, most social media companies are not transparent in how they moderate hate speech \citep{klonick2017new}. So we do not have clear insights into the costs for these companies. There do exist country-specific fines. For example, Germany approved a plan where social media companies can be fined up to 50 million euros if they do not remove hate speech in time \citep{bbc-firms-face-fine-germany}. However, this is location-specific, and it is unclear how this applies to individual cases of hate speech. Defining the value of FP cases is even more difficult. It is unclear how filtering out too much content would affect the company (apart from many annoyed users whose freedom of speech is violated). Therefore, we abstain from estimating the values from the perspective of these companies.

Second, both FP and FN predictions have consequences on the users as the stakeholder. Having too many FP predictions might violate the value of Freedom of Speech since we are filtering out non-hateful posts and, therefore, we cause suppression of free speech. One paper found through a survey that most people think that some form of hate speech moderation is needed, but they also worry about the violation of freedom of speech \citep{olteanu2017limits}. Having too many FN predictions might harm individuals or even result in acts of violence \citep{ecri-hate-speech-and-violence}. Therefore, we need to figure out how we should weigh the values of FP and FN predictions accordingly. We abstain from using time as a unit since it does not make sense to express the consequences of hate speech or the benefits of freedom of speech in time. Therefore, we want to look at the value of freedom of speech and hate speech from an economic perspective. However, we noticed a lack of research in this area. There is one paper where they tried to come up with an economic model for free political speech by looking at the First Amendment to the United States Constitution \citep{posner1986free}. The First Amendment restricts the government from creating laws that could, for example, violate Freedom of Speech \citep{first-amendment-white-house}. The authors explained in \citet{posner1986free} that the lack of research in this area is because most economists do not dive into the legal domain regarding free speech, and free speech legal specialists refrain from doing economic analysis \citep{posner1986free}. The proposed economic model from the paper, for example, includes the cost of harm and the probability that speech results in violence \citep{posner1986free}. However, the authors do not elaborate on how we can define the probability and the costs. Another paper did speculate on this topic by explaining why doing a cost-benefit analysis of free speech is almost impossible \citep{sunstein2018does}. The authors explained that there are too many uncertainties \citep{sunstein2018does}. We can assume that there are values of free speech, but it is too difficult to quantify them \citep{sunstein2018does}. For example, terrorist organizations use free speech to recruit people and call for acts of violence online \citep{sunstein2018does}. At the same time, most other hateful posts will not ever result in actual acts of violence \citep{sunstein2018does}. Therefore, cost values using objective measurements are often case-specific and cannot be defined generically. There is a nonquantifiable risk that acts of violence will happen in the unknown future \citep{sunstein2018does}. But suppose we do know this probability, then there are still too many uncertainties. To calculate the actual costs of hate speech (in our case: to accept the FN predictions), we also need to know the number of lives at risk and how we should quantify the value of each life \citep{sunstein2018does}? The authors claim that analyzing the benefits of free speech is even more difficult \citep{sunstein2018does}. They conclude their work by saying that there are too many problems to empirically evaluate the costs and benefits in the hate speech context \citep{sunstein2018does}.

Therefore, we believe that using objective measurements, such as money, is not realistic for generically expressing the cost values in our project for both stakeholders.

\subsection{Subjective assessment}
- Focus on subjective values of users
- Not companies


\section{Evaluation metrics}
Most hate speech-related studies evaluate their classification methods using standard \textit{machine} metrics such as accuracy, precision, recall, or F1.
%
\citet{rottger2020hatecheck} recognized the shortcomings of these metrics since it is hard to recognize the weak points of classification models.
%
Therefore, the authors presented a suite of 29 carefully selected functional tests to help identify the model's weaknesses \citep{rottger2020hatecheck}.
%
Each test checks different criteria, such as the ability to cope with spelling variations or the ability to detect neutral content that contain slurs \citep{rottger2020hatecheck}.
%
We believe this is a step in the right direction of evaluating and improving hate speech detection models.
%
However, our approach is different since we do not solely focus on evaluating the detection performance of the underlying classification model but instead on evaluating models with a reject option while taking into account the human-perceived value of (in)correct and rejected predictions.
%
Examples include outlining ethical principles of algorithmic systems (Fjeld et al. 2020), developing value-based assessment frameworks (Yurrita et al. 2022), and proposing new metrics for evaluating machine learning systems that incorporate value parameters (Casati, No ̈ el, and Yang 2021)

\todo[inline]{Explain that \cite{olteanu2017limits} claims that we need human-centred metrics instead of abstract metrics such as precision}

\todo[inline]{"proposing new metrics for evaluating machine learning systems that incorporate value parameters \citep{casati2021value}"}
\chapter{Value-sensitive rejection}
As concluded in the \nameref{sec:related-work}, there is a need for \textit{value-sensitive} metrics for measuring the performance of ML models, especially for social-technical applications such as hate speech detection.
%
We also concluded that most automatic hate speech detection methods do not peform well on unseen data.
%
Therefore, in this project, we focus on rejecting ML predictions in a value-sensitive manner.
%
We want to take the values of TP, TN, FP, FN, and rejected predictions into account in deciding when to accept or reject predictions.
%
Correct predictions (TP and TN) result in positive value (gains), while incorrect (FP and FN) and rejected predictions result in negative value (costs).
%
In this section we assume that we know these values, however in section \ref{sec:survey}, we will explain how we assess these values.
%
In this project, we create a confidence metric that is based on the work of \citet{de2000reject}.
%
We create a confidence metric that measures the total value of a ML model for some rejection threshold value based on a set of predictions with their corresponding confidence values and the values of TP, TN, FP, FN, and rejected predictions.
%
The idea of rejecting ML predictions with a confidence metric is that we aim to find the optimal rejection threshold between 0 (accepting all predictions) and 1 (rejecting all predictions).
%
Suppose the optimal confidence rejection threshold is 0.5, then we accept all ML predictions with a confidence that is greater than or equal to 0.5, and reject all ML predictions with a confidence value that is below 0.5.
%
We can determine the optimal rejection threshold by finding the threshold value for which the total value of the ML model with a reject option is maximum.
%
This is a form of a \textit{dependent} rejector.
%
The main benefit is that we can apply the confidence metric to any existing classification model that outputs a prediction label along with a confidence value without retraining the original model.
%
In this section, we explain how we construct our confidence metric and how we will apply it to some of the state-of-the-art hate speech classification models.

\section{Value metric}
\todo[inline]{Explain and proof our value metric and how we use it to measure the total value of a ML model with a reject option}

\section{State-of-the-art}
\subsection{Models}
\todo[inline]{Explain that we are going to use one traditional ML model (Logistic Regression since in related work we found that this got the best performance), one DL model (CNN because of the same reason), and a transformer model (DistilBERT given the recent popularity of transformer models)}

\subsection{Calibration}
\todo[inline]{Explain what model calibration is and why it's necessary}

\subsection{Datasets}
\todo[inline]{Explain that we are going to use the SemEval and Waseem datasets and the concepts of seen and unseen data}

\subsection{Calculating the total value}
\todo[inline]{Explain how we are going to apply our metric to the trained state-of-the-art models to calculate the optimal rejection threshold given the values of TP, TN, FP, FN, and rejection (that we still need estimate)}
\todo[inline]{Explain that we use Kernel Density Estimation to calculate the PDF functions (or stick to prediction counts as in the paper).}

\chapter{Survey study}
\label{ch:survey}
The second part of this research is to find out how we can determine the value ratios between TP, TN, FP, FN, and rejected predictions.
%
We conducted a literature study in section \ref{sec:related-work-value-assessment} and concluded that we want to empirically estimate the value ratios from the perspective of the social media user.
%
In section \ref{sec:me}, we found that ME is a promising technique for estimating these value ratios.
%
Therefore, this chapter discusses how we apply the ME technique in a crowdsourced survey study.
%

%
We design a survey study to ask participants the degree to which they agree or disagree with the decisions of a fictional social media platform called SocialNet.
%
We show the participants different scenarios representing TP, TN, FP, FN, and rejected predictions.
%
The TP and TN scenarios mean that SocialNet successfully detects whether a post is hateful or not, respectively.
%
The FP scenario means that SocialNet incorrectly predicts a non-hateful post as hateful, while the FN scenario implies that SocialNet incorrectly predicts a hateful post as non-hateful.
%
For example, in the FN scenario, the survey shows a hateful post to the participant and explains that SocialNet did not identify the post as hate speech.
%
Then, participants indicate the degree of agreement/disagreement using a scale, and we aggregate the answers per scenario to obtain the value ratios.
%

%
The structure and preparation of our crowdsourced survey study follow the pre-registration plan for social psychology suggested by \citet{van2016pre}.
%
In a pre-registration plan, we describe the hypothesis, procedure, and analysis before conducting the crowdsourced survey study to increase scientific credibility and reproducibility and reduce bias \citep{van2016pre}.
%
It is essential to select the statistical methods for the analysis part beforehand to prevent ourselves from selecting the statistic that best fits the collected data.
%
The content of this chapter reflects the final version of the pre-registration plan created after conducting the pilot survey.
%

%
In section \ref{sec:survey-hypothesis}, we make a hypothesis about the ME method and the value ratios.
%
Section \ref{sec:survey-method} contains all details about the survey setup.
%
Finally, in section \ref{sec:survey-analysis}, we elaborate on the analysis of the survey results.

\section{Hypotheses}
\label{sec:survey-hypothesis}
We listed several hypotheses about the value ratios and the ME method before we conducted the survey experiment.
%
The goal is to reflect on the hypotheses in the discussion to explain why specific results were expected or unexpected.
\begin{itemize}
    \item \textbf{We hypothesize that the values of FP and FN are negative and that the value of an FN is lower than an FP}. We believe that both FP and FN predictions harm social media users; therefore, we think both values should be negative. We believe that allowing hateful content to be publicly visible has a more negative impact on social media users than filtering out neutral content. Therefore, we think an FN's value is lower than an FP's.
    \item \textbf{We hypothesize that the values of TP and TN are both positive and that the value of a TP is greater than a TN.} We believe that both TP and TN predictions positively impact social media users; therefore, we think both values should be positive. We believe predicting hateful content correctly is more valuable to social media users than correctly predicting non-hateful content. Therefore, we think a TP's value is greater than a TN's.
    \item \textbf{We hypothesize that the rejection value is negative and greater than the average value of an FP and an FN.} The critical assumption of using ML models with a reject option is that the negative value of rejection should always be greater than the negative value of an incorrect decision.
    \item \textbf{We hypothesize that FP and FN's absolute magnitudes are greater than TP and TN's.} We believe that social media users find the harm of incorrect predictions more critical than the benefits of correct predictions.
    \item \textbf{We hypothesize that ME is a suitable technique for retrieving the value ratios.} ME seems like a promising technique for retrieving ratio data from judgements about hate speech detection scenarios. We use a 100-level numerical scale for validation. We expect that both scales are correlated and will give similar judgements. Although we also expect the 100-level scale to be suitable for retrieving opinions about the different hate speech detection scenarios, it does not provide the ratio data we need. We also expect that the inter-rater reliability for the 100-level scale will be higher than for the ME scale since the ME scale provides more response freedom. We also expect this since the authors of \citet{roitero2018fine} concluded that the inter-rater reliability of the 100-level scale is higher than the ME scale when rating the relevance of documents.
\end{itemize}

\section{Method}
\label{sec:survey-method}
This section discusses the complete setup of the survey experiment and how we use both scales.

\subsection{Scales}
We use ME as the primary scale of our survey experiment.
%
As we concluded in section \ref{sec:me}, we must also validate the ME scale.
%
We validate the ME scale through cross-modality validation by comparing the results of the ME scale with another scale, as explained in section \ref{sec:me}.
%
The secondary scale is a bounded scale of 100 levels, called the 100-level scale, and we use this scale for four reasons.
%
First, given the limited budget, it is impractical in this project to use other ME scales, such as measuring the intensity of the participants' handgrips to express their judgements.
%
Second, there is no suitable dataset we can use for validation that contains human ratings of different scenarios in hate speech detection.
%
Third, we concluded in \ref{sec:likert} that Likert scales have limited response freedom.
%
Finally, in \citet{roitero2018fine}, the authors concluded that the 100-level scale provides more response freedom than course-grained Likert scales and has several advantages over ME in terms of usability and reliability.
%
The 100-level scale is easier to understand than ME, does not require normalization, and provides more flexibility than a Likert scale \citep{roitero2018fine}.
%
Therefore, we will create two separate surveys with the same scenarios where half of all participants use the 100-level scale and the other half use the ME scale.
%
Both scales are bipolar scales since the participants should be able to either disagree or agree with the scenarios.

\subsection{Normalization}
The ME scale is unbounded and, therefore, provides a lot of response freedom.
%
For example, suppose we first show a scenario, and the participant provides a value (e.g., 100) to indicate the degree of agreement.
%
Suppose we next present a scenario that the participant agrees with more.
%
The participant can always provide a higher value (e.g., 125).
%
However, the results need to be normalized as different participants rate the agreement/disagreement degree differently.
%
As explained in section \ref{sec:me}, we cannot use standard normalization methods such as geometric averaging as we use bipolar scales with negative values.
%
Therefore, we normalize the results by dividing the magnitude estimates of each participant by their maximum estimate.
%
We multiply the normalized magnitude estimates by 100 for the sake of clarity.
%
This way, all magnitude estimates are in the range $[-100, 100]$ while maintaining the ratio properties.

\subsection{Design}
This section lists all independent, dependent, confounding, and control variables analyzed in our experiment.

\subsubsection{Independent variables}
Independent variables are the different hate speech detection scenarios we show to the participants (TP, TN, FP, FN, and rejection).
%
We inform the participants in the case of TP and FP scenarios that SocialNet ranks the hateful post lower on their feed.
%
The users then need to spend more effort finding the post since they need to scroll longer before it becomes visible.
%

%
Initially, in the pilot survey, we explained that detected hateful posts are removed, which could be controversial.
%
Also, we found that participants agreed more with the TP and TN scenarios compared to the degree to which they disagreed with the FP and FN scenarios.
%
Therefore, we decided to explain that hateful posts are ranked lower, that it is expected from detection systems to produce correct predictions, and that incorrect predictions might cause harm to social media users.
%
We did this to prepare the participants to focus on evaluating harm (instead of giving rewards).
%

%
We inform participants in the rejection scenarios that a human moderator needs to check the post (that can be either hateful or not hateful) within 24 hours.
%
Meanwhile, the post remains visible with its original rank on the user's feed.
%
We use 24 hours based on the German NetzDG law, which allows the government to fine social media platforms if they do not remove illegal hate speech within 24 hours \citep{tworek2019analysis}.
%
\begin{itemize}
    \item \textbf{True Positive} Show a hateful post to the user and explain that SocialNet detected hate and ranked the post lower on people's feeds.
    \item \textbf{True Negative} Show a non-hateful post to the user and explain that SocialNet did not detect hate and allowed the post.
    \item \textbf{False Positive} Show a non-hateful post to the user and explain that SocialNet detected hate and ranked the post lower on people's feeds.
    \item \textbf{False Negative} Show a hateful post to the users and explain that SocialNet did not detect hate and allowed the post.
    \item \textbf{Rejection}
          \begin{itemize}
              \item Show a hateful post to the user and explain that SocialNet was uncertain whether the post was hateful or not. An internal moderator will need to check the post within 24 hours. Meanwhile, the post remains visible.
              \item Or show a non-hateful post to the user and explain that SocialNet was uncertain whether the post was hateful or not. An internal moderator will need to check the post within 24 hours. Meanwhile, the post remains visible.
          \end{itemize}
\end{itemize}

\subsubsection{Confounding variables}
\label{sec:survey-confounding-variables}
%
Confounding variables are the different demographic characteristics:
%
\begin{itemize}
    \item \textbf{Nationality} People from different nationalities might have different perceptions and definitions of hate speech and how we should deal with it.
    \item \textbf{Ethnicity} People from different ethnicities might have different perceptions and definitions of hate speech and how we should deal with it.
    \item \textbf{Age} People of different ages might have different perceptions and definitions of hate speech and how we should deal with it.
    \item \textbf{Educational level} People with different educational levels might have different perceptions and definitions of hate speech and how we should deal with it.
    \item \textbf{Sex} According to \citet{gold2018women}, there is no significant difference in how men and women perceive hate. However, we still report sex as a confounding variable since we want to analyze if there are genuinely not any differences.
\end{itemize}

\subsubsection{Control variables}
%
We define two control variables: the measurement scales and the content of the social media posts we show to the participants.
%
We control the measurement scale variable by randomly assigning a participant to use either the 100-level or the ME scale to rate the scenarios.
%
Regarding the scales, as described before, we choose ME as our primary scale and use the 100-level scale for validation.
%
We leave the study of other scales to future work.
%
We control the content of the social media posts in two manners.
%
First, we present all scenarios for all participants randomly to reduce bias.
%
Second, we sample the social media posts for the survey from existing datasets.
% 
We explain the selection procedure in section \ref{sec:data}.
%
\begin{itemize}
    \item \textbf{Scales} The first group of participants must answer the questions using the ME scale. The second group needs to answer the questions using the 100-level scale.
    \item \textbf{Content of the posts} We sample all social media posts from existing datasets and present them to the participants in random order.
\end{itemize}

\subsubsection{Dependent variables}
Our dependent variables are the response values, reliability, validity and the value ratio of TP, TN, FP, FN, and rejection scenarios.
%
\begin{itemize}
    \item \textbf{Response values} All response values the participants give to the different scenarios with either the ME or the 100-level scale.
    \item \textbf{Reliability} Measured using Krippendorff's alpha, where values larger than 0.8 indicate reliable conclusions, and values larger than 0.6 indicate tentative conclusions \citep{krippendorff2004reliability}.
    \item \textbf{Validity} Convergent validity, if two different measures measure the same thing \citep{fitzner2007reliability}. Measured by calculating the correlation between the magnitude estimates and the response values from the 100-level scale.
    \item \textbf{Value ratios of TP, TN, FP, FN, and rejection scenarios} Measured by calculating the median of the normalized magnitude estimate response values of each scenario question and then calculating the mean over the resulting values to come up with the final value for that scenario type.
\end{itemize}

\subsection{Planned sample}
This section discusses how we pick the sample size, recruit the participants, and explains which stopping and exclusion rules we apply.

\subsubsection{Sample size}
There are 4.55 billion active social media users\footnote{\url{https://datareportal.com/reports/digital-2021-october-global-statshot}}.
%
We choose a 90\% Confidence Interval (CI) and 10\% Margin of Error (MoE) for this study.
%
So 90\% of the time, our observations will fall within a 10\% interval \citep{olson2014ways}.
%
According to \citet{olson2014ways}, we need a sample size of 68 participants per survey type to reach the desired CI and MoE values.
%
We choose 10\% MoE since we have a limited budget.
%
We first conduct a pilot survey for 12 participants per scale to gather feedback and check if we need to improve things before the actual experiment.
%
We want to determine the average workload using the pilot survey and whether reducing the MoE by increasing the number of participants is possible.
%
For the pilot survey, we use 24 participants.
%
Therefore, in total we will need $2*12+2*68 = 160$ participants.
%
Of the recruited participants, 50\% identified as female.
%
Half of the participants are assigned the ME scale, and the other half the 100-level scale.

\subsubsection{Participants}
We will use the \href{https://prolific.co}{Prolific} platform for recruiting online participants for the survey study.
%
We will use the following inclusion criteria for our participants:
\begin{itemize}
    \item 18 years of age and older since we show offensive language in the experiment.
    \item Fluent in English.
    \item Approval rating over 90\% on the Prolific platform.
    \item Use one of the following social media platforms regularly (at least once a month): Facebook, Twitter, YouTube, LinkedIn, Pinterest, Google Plus, Tumblr, Instagram, Reddit, VK, Flickr, Vine.co, Meetup, ask.fm, Snapchat, TikTok, Medium.
\end{itemize}
%
Every participant will be paid based on the hourly wage of 9.0 GBP (about 10,67 Euro), indicated as good pay by the platform\footnote{\url{https://prolific.co/pricing}}.
%
We use the following exclusion/rejection criteria:
\begin{itemize}
    \item Participants who fail the two attention checks. We will include two Instructional Manipulation Checks to check if the user pays attention to the survey\footnote{\url{https://researcher-help.prolific.co/hc/en-gb/articles/360009223553}}.
    \item Participants who do not complete all questions.
    \item Participants who disagree with the informed consent before the start of the survey. We are not allowed to collect and process their data if they do not consent.
    \item Participants who disagree with the informed consent before the survey start. We are not allowed to collect and process their data if they do not consent.
\end{itemize}
%
We select a balanced set of participants in Prolific, among which 50\% are men and 50\% are women.

\subsection{Data}
\label{sec:data}
Depending on the assigned survey group, all subjects must judge several TP, TN, FP, FN, and rejection scenarios using either the ME or the 100-level scale.
%
We select the posts used in the scenarios from a public dataset \citep{basile2019semeval} that contains 13,000 English tweets.
%
Each tweet is annotated with three categories: hate speech (yes/no), target (generic group or an individual), and aggressiveness (yes/no).
%
Therefore, we have one neutral and four groups of hateful tweets: generic target + aggressive, individual target + aggressive, generic target + non-aggressive, and individual target + aggressive.
%
For the rejection scenarios, we need both neutral and hateful tweets.
%
Therefore, we need at least eight tweets per scenario type (TP, TN, FP, FN, and rejection).
%
We need 40 tweets, where 20 are hateful, and 20 are not hateful, to create 40 different scenarios.
%

%
We want to select the most representative tweets from the dataset.
%
Randomly selecting the tweets from the dataset is insufficient as the dataset might contain sample retrieval bias, as explained in section \ref{sec:related-work-challenges}.
%
We might retrieve too many similar tweets about the same topic when randomly selecting the tweets.
%
Therefore, we perform content analysis to create a selection of tweets that is as representative and diverse as possible.
%
We provide an overview of our selection process in figure \ref{fig:clustering}.
%

%
We exclude all tweets that contain Twitter replies and mentions since they have unclear contexts.
%
Then we preprocess all tweets by removing the URLs and hashtags.
%
Finally, we use clustering analysis to select 40 tweets for our study.
%
We perform latent semantic analysis (LSA) and k-means clustering on each group of tweets.
%

%
We use the term frequency-inverse document frequency (TF-IDF) to represent all documents and their words, also known as terms, in a matrix where the term frequencies indicate how important that term is to the document \citep{aggarwal2012survey}.
%
The term frequencies are multiplied by the inverse document frequency so that terms that often occur in all documents, such as stop words, will end up with a lower value in the matrix \citep{aggarwal2012survey}.
%

%
Then, we use singular value decomposition (SVD) for dimensionality reduction to transform the output matrix of the TF-IDF step.
%
The transformed matrix is more suitable for text clustering techniques since documents with similar terms are now grouped \citep{aggarwal2012survey}.
%
The combination of TF-IDF and SVD is also known as LSA and is suitable for clustering purposes \citep{aggarwal2012survey}.
%

%
Finally, we apply the unsupervised learning technique k-means to the output of the LSA method to cluster all tweets into k clusters.
%
We calculate the silhouette coefficient to determine the optimal cluster size (k value) for the neutral tweets and the four groups of hateful tweets.
%
The silhouette analysis indicates setting k as large as possible.
%

%
We select the five nearest data samples to each cluster centroid.
%
From this selection, we manually choose one tweet per cluster using a majority vote from three group members to create the final set of 40 tweets.
%
Based on the silhouette coefficient, we use a cluster size of 20 for the neutral tweets and select one tweet per cluster to collect 20 neutral tweets.
%
Furthermore, we use a cluster size of 5 for each group of hateful tweets to collect 20 hateful tweets.
%

%
Refer to appendix \ref{sec:appendix-scenarios} for the resulting list of all scenarios.
%

\begin{figure}
    \centering
    \includegraphics[scale=.75]{Figures/clustering.pdf}
    \caption{Flow diagram that visualizes how we perform content analysis to cluster and select the tweets for our survey study.}
    \label{fig:clustering}
\end{figure}

\subsection{Procedure}
In figure \ref{fig:survey-procedure}, we present the procedure of the two surveys, one where participants use the ME scale and another where participants use the 100-level scale.
%
We use LimeSurvey\footnote{\url{https://www.limesurvey.org/}} as our survey tool.
%
The survey first presents the informed consent policy and excludes participants that do not agree with it.
%
Next, we show introductory texts to the participants to explain what we will expect from them and to explain the structure of the survey.
%
Using the ME scale, we first present a training phase where the participants need to estimate five different line lengths using any positive value to get familiar with using the ME scale.
%
Then, we randomly present two attention checks and 40 scenarios representing the TP, TN, FP, FN, and rejection scenarios (with eight scenarios per type).
%
Each scenario contains several questions with the same structure.
%
The first question is whether participants think the post is hateful (yes/no).
%
The second question is whether participants agree, disagree, or are neutral with SocialNet's decision.
%
In the case of nonneutral, we ask a third question about the degree to which participants agree or disagree with the machine's decisions, using either the ME or 100-level scale, depending on their group.
%
There is no time limit for answering the questions, and all data is anonymous.
%
Finally, we will inform the participants not to put identifiers in their answers.
%
Refer to Appendix \ref{sec:appendix} for all presentation texts, the informed consent, and some scenario examples.

\begin{figure}
    \centering
    \includegraphics[scale=.75]{Figures/survey.pdf}
    \caption{Flow diagram that visualizes the procedure of the survey study. We assign half of the participants to the ME survey and the other half to the 100-level survey.}
    \label{fig:survey-procedure}
\end{figure}

\section{Analysis}
\label{sec:survey-analysis}
First, we calculate the value ratios of the TP, TN, FP, FN, and rejection scenarios in hate speech detection using the survey's results.
%
Second, we analyze the quality of our survey method by looking at two aspects: reliability and validity.

\subsection{Value ratios}
\label{sec:analysis-values}
The survey study aims to determine the value ratios of the TP, TN, FP, FN, and rejection scenarios in the context of hate speech detection.
%
The metric from section \ref{sec:value-metric} takes these numerical values as input to calculate the optimal rejection threshold.
%
We do not need to know the absolute values but only the relative values.
%
For example, if we set all values to 1, we retrieve the same optimal rejection threshold as setting all values to 1000.
%
We use a bipolar scale for question 3 in the survey since we ask the participants the degree to which they agree, disagree, or are neutral with the decision of SocialNet.
%
For both scales, we will convert disagreement values to negative values, neutral values to 0, and agreement values to positive values.
%
Since we found that the data of both scales is skewed after conducting the pilot survey, we first apply the median to the individual questions' results.
%
Then we calculate the mean value over the resulting values to retrieve the final aggregated value ratios.
%
For example, to calculate the aggregated $V_{tp}$ values for both scales, we use:
\begin{align*}
    V_{tp}^{ME} = \frac{1}{n} \sum_{i=1}^{n} r_{i, tp}^{ME}   & \quad  \parbox{35em}{\footnotesize where $n$ is the total number of all participants for TP scenarios and $r_{i, tp}^{ME}$ is the  \\median response value of TP question number $i$ rated with the ME scale.}\\
    V_{tp}^{100} = \frac{1}{n} \sum_{i=1}^{n} r_{i, tp}^{100} & \quad  \parbox{35em}{\footnotesize where $n$ is the total number of all participants for TP scenarios and $r_{i, tp}^{100}$ is the \\median response value of TP question number $i$ rated with the 100-level scale.}
\end{align*}
%
We apply the same calculations for the remaining scenario types.
%
The results should give us an understanding of how the participants feel towards the different scenarios: TP, TN, FP, FP, and rejection.
%
We define the value ratios we need for the metric using the aggregated values of the TP, TN, FP, FN, and rejection scenarios rated with the ME scale since the ME scale provides us with ratio data.
%
We will not use the aggregated values of the 100-level scale for our metric since the 100-level scale does not provide ratio data, but we will still present them.

\subsection{Reliability}
\label{sec:reliability}
Reliability is about whether we can trust our results and if we get consistent results \citep{fitzner2007reliability}.
%
We do this by mainly looking at inter-rater reliability.
%
Different participants should give approximately the same judgements to the same scenarios.
%
We measure the inter-rater reliability using Krippendorff's alpha \citep{maddalena2017crowdsourcing, krippendorff2004reliability}.
%
We calculate the inter-rater reliability value for the complete survey's data for the normalized ME and 100-level values.
%
We use the inter-rater reliability scores to compare the ME scale with the 100-level scale.
%
We also separately study the inter-rater reliability values for the different types of scenarios (TP, TN, FN, FP, and rejection).
%
This experiment does not consider other types of reliability, such as test-retest reliability.
%
Guaranteeing test-retest reliability would require us to redo the complete experiment at a different time for the same participants, which is infeasible for this project, given the limited time and budget.

\subsection{Validity}
\label{sec:analysis-validity}
Validity is about whether we are measuring the things we want to measure \citep{fitzner2007reliability}.
%
The main goal of this aspect is to validate if we can use the ME technique to measure participants' opinions about hate speech detection scenarios.
%
There are multiple types of validity, but we focus mainly on convergent validity (part of construct validity), content validity, and face validity \citep{fitzner2007reliability}.
%
Construct validity checks whether there is an agreement between a theory and a measurement device or procedure \citep{fitzner2007reliability}.
%
Convergent validity is about the correlation between different measures to see if they measure the same phenomenon \citep{fitzner2007reliability}.
%
Content validity is about letting experts review the proposed research questions and procedure \citep{fitzner2007reliability}.
%
Face validity is a subjective type of validity, and it is about why we think the questions and proposed procedures are valid \citep{fitzner2007reliability}.
%

%
We analyze convergent validity by performing cross-modality validation.
%
Following the approach from \citet{roitero2018fine}, we analyze the correlation between the ME scale and the 100-level scale.
%
We can verify that they measure the same phenomenon if we find that both scales are positively correlated.
%
However, we can also expect a low correlation since the ME scale is a (normalized) unbounded scale, and the 100-level scale is bounded.
%
Nevertheless, we think both scales will give similar results, meaning that high ME responses should correspond to high 100-level scale responses and low ME responses to low 100-level scale responses.
%
To guarantee content validity, we let experts (the supervisors of this thesis project) check the pre-registration report before conducting the experiments.
%
We tackled face validity in section \ref{sec:related-work-value-assessment} by arguing why we think the ME technique is suitable for measuring people's opinions about hate speech detection scenarios.
%
We exclude other forms of validity from this experiment because they are irrelevant or infeasible.
%
For example, external validity is about the degree to which the findings can be generalized to other settings or groups \citep{fitzner2007reliability}.
%
We think people with different demographic characteristics perceive hate speech differently since people have other norms and values.
%
We believe that if we conduct this experiment using different groups of participants, we might retrieve different value ratios.
%
Therefore, we decided not to create too many participant inclusion criteria but take a random sample of global social media users.
%
We would have to experiment with multiple groups with different demographic characteristics to guarantee external validity.
%
We left this for future work to investigate in full detail.
%
However, we still try to analyze if we can find any differences between participants with different demographic characteristics in the dataset we retrieve (refer to section \ref{sec:analysis-demographic}).

\subsection{Demographics}
\label{sec:analysis-demographic}
As we conduct the survey study only once for a group of participants, among which 50\% are men and 50\% are women, the remaining demographic characteristics can be quite diverse.
%
Nevertheless, we verify whether there are any significant statistical differences between groups of participants with different demographic characteristics.
%
We expect that demographic characteristics influence people's perception of hate speech and how we should deal with it.
%
Therefore, we apply several statistics to the results of each scenario to analyze if we can find differences between different demographic groups.
%

%
Prolific provides information about the demographic characteristics of the participants, out of which we analyze six features based on what Prolific provides us in the survey results: sex, student (whether they are still a student or not), continent, nationality, language, and ethnicity.
%
We manually add the continent feature based on the values of the nationality feature.
%
Most features overlap with our pre-defined confounding variables from section \ref{sec:survey-confounding-variables}, where features such as nationality, continent, and language are highly correlated.
%
We exclude age as almost all participants fall between 20 and 30 years old.
%

%
We have multiple groups (more than two) for nationality, ethnicity, and language and two groups for the features student, sex, and continent (since we found only two continents in the demographic data of all participants).
%
We apply either analysis of variance (ANOVA) (parametric) or Kruskal-Wallis (non-parametric) when we have more than two groups.
%
Furthermore, we apply an unpaired two-sample t-test (parametric) or the Mann-Whitney U Test (non-parametric) when we have exactly two groups.
%

%
First, we check if we can apply the parametric statistics by checking if their assumptions hold in our dataset.
%
If not, then we use the non-parametric tests.
%
We apply ANOVA and the t-test when the data meets the following three conditions: homogeneity of variance (each population has the same variance), normality (normal distribution of the error), and independence (the observations are independent of each other) \citep{howell2012statistical}.
%
We use Bartlett's test of homogeneity of variances and the Shapiro-Wilk test of normality to check if we can apply ANOVA and the t-test.
%
We obey the independence condition since we collect the data of all survey study participants independently.
%

%
ANOVA and the t-test can be robust to violations of the homogeneity of variances and the normality assumptions \citep{howell2012statistical}.
%
However, if one of the assumptions is violated, then it is essential to keep the sample sizes as equal as possible \citep{howell2012statistical}.
%

%
Finally, for the multi-group features (nationality, language, and ethnicity), we apply pairwise statistical tests (Mann-Whitney U or t-test) between all groups.
%
We only do this for the scenarios where we find significant differences between the groups through ANOVA/Kruskal-Wallis.
%

%
However, we now may introduce Type I errors as we perform many pairwise statistical tests between all groups.
%
As a result, we might incorrectly reject the null hypothesis for some pairwise tests, meaning that we find significant differences between some groups while there are none.
%
Therefore, we perform the post hoc Benjamini-Hochberg procedure to correct the p-values of the pairwise test results to control the Type I errors.
\chapter{Results}
This chapter includes the results of the survey study (chapter \ref{ch:survey}) and the experiments with the value-sensitive rejector (chapter \ref{ch:rejector}).
%
We first present the results of the survey study as the experiments with the value-sensitive rejector depended on the outcomes of the survey study.
%

%
The goal of the survey study was to retrieve the value ratios of TP, TN, FP, FN, and rejected predictions in hate speech detection from the user's perspective.
%
We retrieved the value ratios using a scale called Magnitude Estimation (ME) and validated the ME scale by conducting a separate survey that uses a bounded scale that consists of 100 levels, called the 100-level scale.
%
The goals of the experiments with the value-sensitive rejector were finding out how the rejector behaves on different models and datasets, if ML with a reject option is beneficial for hate speech detection, and comparing the value-sensitive metric against machine metrics such as accuracy.
%

%
In section \ref{sec:results-survey-study}, we present the results of the complete  survey study that we gathered after conducting the pilot survey, and in section \ref{sec:results-rejector}, we present the results of the experiments with the value-sensitive rejector.

\section{Survey study}
\label{sec:results-survey-study}
This section presents the value ratios in section \ref{sec:results-value-ratios}, the reliability analysis in section \ref{sec:results-reliability}, the validity analysis in section \ref{sec:results-validity}, and the demographic analysis in section \ref{sec:results-demographics}.
%


\subsection{Value ratios}
\label{sec:results-value-ratios}
As explained in section \ref{sec:analysis-values}, we first calculated the median of all responses for each question.
%
We use the median since we found in the pilot survey that the data from both scales are highly skewed with outliers.
%
Then, we calculate the mean of all values with the same scenario type (TP, TN, FP, FN, or rejection).
%
We show the resulting value ratios in table \ref{tab:values-reliability}.
%
We

\begin{table}[t]
    \centering
    \begin{tabular}{lcccc}
        \toprule
                           & \multicolumn{2}{c}{\textbf{ME}} & \multicolumn{2}{c}{\textbf{100-level}}                                        \\
        \cmidrule(l){2-3} \cmidrule(l){4-5}
                           & $\boldsymbol{\alpha}$           & $\textbf{v}$                           & $\boldsymbol{\alpha}$ & $\textbf{v}$ \\
        \midrule
        \textbf{TP}        & 0.07                            & 18.15                                  & 0.04                  & 77.00        \\
        \textbf{TN}        & 0.10                            & 36.32                                  & 0.11                  & 86.31        \\
        \textbf{FP}        & 0.39                            & -16.69                                 & 0.07                  & -51.00       \\
        \textbf{FN}        & 0.92                            & -28.08                                 & 0.14                  & -62.43       \\
        \textbf{Rejection} & -0.31                           & -4.82                                  & 0.07                  & -16.37       \\
        \midrule
        \textbf{All}       & 0.78                            & ---                                    & 0.44                  & ---          \\
        \bottomrule
    \end{tabular}
    \caption{Krippendorff's alpha ($\alpha$) and the scenario values ($v$) for TP, TN, FP, FN, and rejection scenarios for both scales: the Magnitude Estimation (ME) and 100-level scale.}
    \label{tab:values-reliability}
\end{table}

\subsection{Reliability}
\label{sec:results-reliability}

\subsection{Validity}
\label{sec:results-validity}

\subsection{Demographics}
\label{sec:results-demographics}

\section{Value-sensitive rejection}
\label{sec:results-rejector}
\todo[inline]{Explain that the metric conditions still hold if we set $V_{tp}$ and $V_{tn}$ to 0}

\chapter{Discussion}
In this thesis project, we worked on a hybrid human-AI solution for detecting hate speech.
%
The main problem is that manual moderation is the most reliable but infeasible, and that automatic moderation through detection algorithms is the most performant but sometimes unreliable.
%
We focused on rejecting predictions of ML models for hate speech detection.
%
However, determing when to accept or reject predictions depends on the context and, more specifically, the implications of accepting/rejecting correct or incorrect predictions.
%
We denoted these implications as values of TP, TN, FP, FN, and rejected predictions.
%
Our main goal was finding out how we can reject ML predictions in a value-sensitive manner for hate speech detection.
%
We split this up into two parts.
%
First, we wanted to find out how we can measure the total value of ML models with a reject option.
%
By maximizing the total value, we know when we need to accept or reject predictions.
%
We tackled this by introducing a value-sensitive metric that we use for calculating the optimal rejection threshold.
%
Second, we wanted to determine the value ratios between TP, TN, FP, FN, and rejected predictions.
%
We determined these value ratios by conducting a large survey study in which we presented participants with different hate speech detection scenarios in which they had to provide their judgements using the ME scale.
%

%
This chapter analyzes the results from chapter \ref{ch:results}.
%
First, we discuss the main findings of the survey study in section \ref{sec:discussion-survey} and the main findings of our value-sensitive rejector in section \ref{sec:discussion-rejection}.
%
Then, we answer the our research questions in section \ref{sec:discussion-research}.
%
Finally, we highlight some limitations of our approach in section \ref{sec:discussion-limitations} and give some recommendations in section \ref{sec:discussion-recommendations}.


\section{Survey study}
\label{sec:discussion-survey}
\todo[inline]{Discuss value ratios}
\todo[inline]{Discuss reliability values and why the values are low when looking at the scenario types only}
\todo[inline]{Discuss why TN3 and TN7 have highest agreement scores}
\todo[inline]{Discuss why FN3 and FN7 have highest disagreement scores}
\todo[inline]{Discuss why FP7 and REJ4 have most sig. differences among all features. And why TP6, FN5, and REJ1 have the second most differences. Not clear why FP7. But REJ4 is a neutral tweet about a political topic. People could have different opinions about refugees espcially from different nationalities. This might explain the differences.}
\todo[inline]{Only FN5 has the most pairwise sig. differences, which is also a tweet about the building the wall.}
\todo[inline]{However, when looking at the other scenarios, we mainly see that there are no differences between groups with different demographic features.}
\todo[inline]{For the sex feature, we do not find any differences between the men and women. This is what we expected as related work also found this (refer to work).}
\todo[inline]{Furthermore, for the features with multiple groups (more than two) where we did find sig. differences, we often do not find any pairwise sig. differences between the groups.}
\todo[inline]{Therefore, we can conclude that for our dataset, people with different demographic characterisitcs overall tend to give the same judgements.}
\todo[inline]{However, we can see that differences between groups of participants for features such as nationality, language, or ethnicity are greater than features such as sex or student.}
\todo[inline]{The results of both the Kruskal-Wallis and the pairwise Mann-Whitney U tests are very similar between the nationality and language features. This can be explained by the fact that the groups of both features are very similar as only a couple of participants differ between the two features.}
\todo[inline]{Also, we see that there are more hateful posts (10) than non-hateful ones (5) with significant differences, meaning that hateful tweets are more sensitive to trigger differences between groups with different demographic characterisitcs.}
\todo[inline]{Also, we should keep in mind that because of the clustering analysis, we created a selection of tweets as diverse as possible. So some posts may not result in any differences between groups while others do. For example, there are five posts, both hateful and non-hateful, about building a wall in America across the Mexican border and four posts where there is at least one feature with significant differences between the groups.}
\todo[inline]{Therefore, we can conclude that the significant differences between groups for a scenario depends highly on the content of the social media post.}

\section{Value-sensitive rejection}
\label{sec:discussion-rejection}
\section{Implications}
\todo[inline]{Explain that \cite{olteanu2017limits} claims that we need more human-centred metrics instead of abstract metrics such as precision and we agree with that by introducing our own human-centred metric}

\section{Research questions}
\label{sec:discussion-research}

\todo[inline]{Answer research questions}

\section{Limitations}
\todo[inline]{Hate speech is difficult domain as there tend to be a lot of disagreement between people about what is considered hate speech and what not. \citet{ross2017measuring} found low Krippendorff alpha values in a hate speech survey. So our findings are in line with theirs.}
\todo[inline]{Explain limitations of the metric and the survey study}
\todo[inline]{Finally, we should point out the limitations of the demographic analysis.}
\todo[inline]{First, the sample sizes of the demographic groups are not representative of the population of those groups.}
\todo[inline]{For example, for the nationality feature, we had five participants from Spain.}
\todo[inline]{Second, we have to keep in mind that some features where we did find sig. differences might have happened by chance. There may have been participants that did not understood the scenario, either because the lack of english or because they rushed through the survey.}

\todo[inline]{The rejection threshold is calculated using the test set. This test set needs to be as realistic as possible. Furthermore we need to have calibrated models since we rely purely on the confidence values. This is also hard to realize. Temperature scaling can help, but it is still limited.}

\section{Recommendations}
\todo[inline]{Magnitude Estimation seems promising for future research in HCI.}
\todo[inline]{Personal and demographic characterisitcs might have  a big impact. So further analysis on those aspects seem relevant.}
\todo[inline]{Perhaps we can train ML models using the values of TP, TN, FP, FN, rejection in an integrated rejector. So we train the ML model and the rejector simultanously using the values from the survey. So then during training, the FN predictions are punished more than FP predictions.}
\chapter{Conclusion}
In this thesis project, we worked on a hybrid human-AI solution for detecting hate speech.
%
The main problem is that manual moderation is the most reliable solution but simply infeasible, and that automatic moderation through detection algorithms is the most performant but sometimes unreliable.
%
Therefore, we focused on rejecting predictions of ML models for hate speech detection.
%
However, determing when to accept or reject predictions depends on the context and, more specifically, the implications of accepting/rejecting correct or incorrect predictions.
%
We denoted these implications as values of TP, TN, FP, FN, and rejected predictions.
%
Our main goal was finding out how we can reject ML predictions in a value-sensitive manner for hate speech detection.
%
We split this up into two parts.
%
First, we wanted to find out how we could measure the total value of ML models with a reject option.
%
By maximizing the total value, we know when to accept or reject predictions.
%
% We tackled this by introducing a value-sensitive metric that we use for calculating the optimal rejection threshold.
%
Second, we wanted to determine the value ratios between TP, TN, FP, FN, and rejected predictions.
%
% We determined these value ratios by conducting a large survey study in which we presented participants with different hate speech detection scenarios that they had to judge using the ME scale.
%
% By combining the two parts, we presented an intuitive hybrid human-AI solution for detecting hate speech in a value-sensitive manner.


%----------------------------------------------------------------------------------------
%	THESIS CONTENT - APPENDICES
%----------------------------------------------------------------------------------------

\appendix % Cue to tell LaTeX that the following "chapters" are Appendices

% Include the appendices of the thesis as separate files from the Appendices folder
% Uncomment the lines as you write the Appendices

\chapter{Survey}
\label{sec:appendix}
This appendix contains all the presentation material of the survey: the consent, explanation texts, and some examples of scenarios.
\section{Consent}
\begin{flushleft}
    You are being invited to participate in a research study titled "Costs of predictions in hate speech detection". This study is being done by Philippe Lammerts from the TU Delft.
\end{flushleft}
\begin{flushleft}
    The purpose of this research study is to find out what social media users think of different scenarios of hate speech detection on social media. It will take you approximately 22 minutes to complete. These scenarios consist of two things. First, we show a specific social media post that can be either hateful or not hateful. You need to indicate if you feel that the post is hateful or not. Second,  we explain how the social media platform dealt with this post. You need to indicate whether you agree/disagree/are neutral about the platform's decision. The results of the survey will be used in my thesis.
\end{flushleft}
\begin{flushleft}
    As with any online activity, the risk of a breach is always possible. To the best of our ability, your answers in this study will remain confidential. We will minimize any risks by making this survey completely anonymous. Therefore, please do not provide any personal information anywhere. The anonymous results might be shared publicly in the future.
\end{flushleft}
\begin{flushleft}
    Your participation in this study is entirely voluntary, and you can withdraw at any time.
\end{flushleft}
\begin{flushleft}
    Warning: some of the scenarios used in this experiment contain harmful and offensive content that may make some people feel uncomfortable.
\end{flushleft}
\begin{flushleft}
    Feel free to contact me with any questions or feedback you might have:
    p.m.lammerts@student.tudelft.nl
\end{flushleft}

\section{Introduction}
\subsection{Short introduction ME}
\begin{itemize}
    \item You will be presented with a series of different scenarios.
    \item For each scenario, you need to answer two questions.
    \item We will explain the exact instructions later.
    \item But first, we will let you familiarize yourself with a scale called Magnitude Estimation.
\end{itemize}

\subsection{Short introduction 100}
\begin{itemize}
    \item You will be presented with a series of different scenarios.
    \item For each scenario, you need to answer two questions.
    \item We will explain the exact instructions in the next page.
\end{itemize}


\subsection{Introduction}
You will be presented with a series of different scenarios.
\begin{itemize}
    \item Each scenario describes a situation of a social media user who wants to post a specific message on a fictional social media platform we now call SocialNet.
    \item These posts can be neutral or contain hateful content.
    \item SocialNet uses automated detection systems for detecting hate speech.
    \item When doing the study, you should be aware that it is expected for SocialNet to correctly classify hate speech. Wrong classifications are undesirable as they may cause harm to people.
\end{itemize}

\begin{flushleft}
    Each scenario describes one of the following situations for a specific social media post:
\end{flushleft}

\begin{enumerate}
    \item \textbf{You are a user of the SocialNet platform and have \textbf{not} seen this post on your main feed because SocialNet's automated detection system is confident that it is hateful.}
          \begin{itemize}
              \item You can still find this post when you scroll down your feed since SocialNet ranks hateful posts lower.
              \item If the post is not hateful after all, then the detection system was incorrect. This neutral post is now ranked lower on people's feeds with the consequence that the post cannot easily reach the author's followers.
              \item If the post is indeed hateful, then the detection system was correct.
          \end{itemize}
    \item \textbf{You are a user of the SocialNet platform and just saw this post on your main feed because SocialNet's automated detection system is confident that it is \textbf{not hateful}.}
          \begin{itemize}
              \item This post remains visible on other people's main feeds as well.
              \item If the post is hateful after all, then the detection system was incorrect. This hateful post is now visible on people's main feeds with the consequence that they can get harmed.
              \item If the post is indeed not hateful, then the detection system was correct.
          \end{itemize}
    \item \textbf{You are a user of the SocialNet platform and just saw this post on your main feed because SocialNet's automated detection system was not confident enough in whether it was hateful or not.}
          \begin{itemize}
              \item An internal human moderator at SocialNet needs to look at it within at most 24 hours.
              \item Meanwhile, the post remains visible on people's main feeds.
          \end{itemize}
\end{enumerate}

\section{Scales}
\subsection{100-level scale explanation}
For each scenario, you need to answer two questions:
\begin{enumerate}
    \item First, you need to indicate whether you feel that this post is hateful or not hateful.
    \item Second, your task is to tell how you feel about SocialNet's decision.
          \begin{itemize}
              \item If you feel neutral about SocialNet's decision, this value will be equal to 0.
              \item If you (dis)agree with the decision, you need to indicate how much you (dis)agree by assigning any number between 1 and 100.
              \item A large number means you (dis)agree with it a lot, while a small number means you (dis)agree with it a little.
              \item Try to make each number match the intensity as you perceive it.
          \end{itemize}
\end{enumerate}

\begin{flushleft}
    Don't worry, we will provide the same explanations in the questions as well.
\end{flushleft}

\subsection{ME scale explanation}
The following text is based on the survey setup from \citet{moskowitz1977magnitude}.\\

\begin{flushleft}
    For each scenario, you need to answer two questions:
\end{flushleft}

\begin{enumerate}
    \item First, you need to indicate whether you feel that this post is hateful or not hateful.
    \item Second, your task is to tell how you feel about SocialNet's decision.
          \begin{itemize}
              \item If you feel neutral about SocialNet's decision, this value will be equal to 0.
              \item If you (dis)agree with the decision from SocialNet, you need to assign any number that is greater or equal to 0 that reflects how much you (dis)agree with the decision.
              \item Assign any number that seems appropriate to you.
              \item A large number means you (dis)agree a lot, while a small number means you (dis)agree a little.
              \item If you (dis)agree twice as much with the current decision as with the previous one, you need to assign a number that is twice as large as the previous number.
              \item Or, if you (dis)agree half as much with the current decision as with the previous one, you need to assign a number that is half as large as the previous number.
              \item You can use any number or decimal you want, but make each assignment proportional to your subjective impression.
              \item Please try to not only use appropriate numbers but also avoid restricting your choice of numbers from 1 to 10.
              \item Try to remember which values you used before and to compare succeeding scenarios with each other.
          \end{itemize}
\end{enumerate}

\begin{flushleft}
    Don't worry, we will provide the same explanations in the questions as well.
\end{flushleft}

\section{Training phase ME}
``As a warm-up task, to familiarize you with magnitude estimation, you will be shown a sequence of five lines, one at a time.
\begin{itemize}
    \item For each line, enter a number into the text box below the displayed line. This number should reflect your perception of the length of the line. You may use any numbers that seem appropriate to you—whole numbers or decimals. However, you may not use negative numbers or zero.
    \item For each subsequent line, enter a number that reflects your perception of its length, relative to the previous line. For example, if you feel that the current line is twice as long as the previous, then you should assign a number that is twice as large as the number you used previously.
\end{itemize}

\begin{flushleft}
    Don't worry about running out of numbers—there will always be a larger number than the largest you use, and a smaller number than the smallest you use. Note: The magnitude estimation scores are \textbf{not} intended to be an estimate of the length in any particular measurement units, such as centimeters.'' \citep{maddalena2017crowdsourcing}
\end{flushleft}

\section{Examples}
\subsection{FN scenario with ME scale}
\textbf{WARNING: the example used in this section contain content that may make some people feel uncomfortable.}

\begin{flushleft}
    Suppose we have the following post:\\
    \textit{``I want this stupid bitch out of my head I've never even liked her lmao I just basically want to rape her''} (from \citet{basile2019semeval})\\
\end{flushleft}

\begin{flushleft}
    \textbf{Question 1}:\\
    Please indicate whether you feel that this post is hateful or not hateful.\\

    **Select input with options: 'hateful' or 'not hateful'**\\
\end{flushleft}

\begin{flushleft}
    \textbf{Question 2}:\\
    You are a user of the SocialNet platform and just saw this post on your main feed because SocialNet's automated detection system is confident that it is not hateful.
\end{flushleft}

\begin{itemize}
    \item This post remains visible on other people's main feeds as well.
    \item If the post is hateful after all, then the detection system was incorrect. This hateful post is now visible on people's main feeds with the consequence that they can get harmed.
    \item If the post is indeed not hateful, then the detection system was correct.
    \item You should be aware that it is expected for SocialNet to correctly classify hate speech. Wrong classifications are undesirable as they may cause harm to people.
\end{itemize}

\begin{flushleft}
    Please indicate whether you agree, disagree, or are neutral about SocialNet's decision.\\

    **Select input with options: 'agree', 'neutral', or 'disagree'.**\\
\end{flushleft}

\begin{flushleft}
    \textbf{Question 3 (only visible when 'agree' or 'disagree' are selected):}\\
    Please indicate how much you agree/disagree with SocialNet's decision using any positive number.\\

    **Show ME input field that allows all positive values greater than 0.**
\end{flushleft}

\subsection{FP scenario with 100-level scale}
\begin{flushleft}
    Suppose we have the following post:

    \textit{``Ontario concerns were not listened at immigration meeting in Ottawa For illegal crossing''} (from \citet{basile2019semeval})\\
\end{flushleft}

\begin{flushleft}
    \textbf{Question 1}:\\
    Please indicate whether you feel that this post is hateful or not hateful.\\

    **Select input with options: 'hateful' or 'not hateful'**\\
\end{flushleft}

\begin{flushleft}
    \textbf{Question 2}:\\
    You are a user of the SocialNet platform and have not seen this post on your main feed because SocialNet's automated detection system is confident that it is hateful.
\end{flushleft}

\begin{itemize}
    \item You can still find this post when you scroll down your feed since SocialNet ranks hateful posts lower.
    \item If the post is not hateful after all, then the detection system was incorrect. This neutral post is now ranked lower on people's feeds with the consequence that the post cannot easily reach the author's followers.
    \item If the post is indeed hateful, then the detection system was correct.
    \item You should be aware that it is expected for SocialNet to correctly classify hate speech. Wrong classifications are undesirable as they may cause harm to people.
\end{itemize}

\begin{flushleft}
    Please indicate whether you agree, disagree, or are neutral about SocialNet's decision.

    **Select input with options: 'agree', 'neutral', or 'disagree'.**\\
\end{flushleft}

\begin{flushleft}
    \textbf{Question 3 (only visible when 'agree' or 'disagree' are selected):}\\
    Please indicate how much you agree/disagree with SocialNet's decision using any positive number from 1 to 100. If you feel neutral about SocialNet's decision, select neutral in the field above.


    **Show a numerical slider with values between 1 and 100.**\\
\end{flushleft}


\subsection{Rejection scenario with 100-level scale}
\begin{flushleft}
    Suppose we have the following post:

    \textit{``Ever been so hungover that your stomach feels like it's eating itself''} (from \citet{basile2019semeval})\\
\end{flushleft}

\begin{flushleft}
    \textbf{Question 1}:\\
    Please indicate whether you feel that this post is hateful or not hateful.\\

    **Select input with options: 'hateful' or 'not hateful'**\\
\end{flushleft}

\begin{flushleft}
    \textbf{Question 2}:\\
    You are a user of the SocialNet platform and just saw this post on your main feed because SocialNet's automated detection system was not confident enough in whether it was hateful or not.
\end{flushleft}
\begin{itemize}
    \item An internal human moderator at SocialNet needs to look at it within at most 24 hours.
    \item Meanwhile, the post remains visible on people's main feeds.
\end{itemize}

\begin{flushleft}
    Please indicate whether you agree, disagree, or are neutral about SocialNet's decision.\\

    **Select input with options: 'agree', 'neutral', or 'disagree'.**\\
\end{flushleft}

\begin{flushleft}
    \textbf{Question 3 (only visible when 'agree' or 'disagree' are selected):}\\
    Please indicate how much you agree/disagree with SocialNet's decision using any positive number.\\

    **Show a numerical slider with values between 1 and 100.**\\
\end{flushleft}

\chapter{Results}
\label{sec:appendix-b}
This appendix contains the remaining results of the demographic analysis on the survey's results from section \ref{sec:results-demographics} and the experiments with the value-sensitive rejector from section \ref{sec:results-rejector}.

\section{Demographic analysis}

\begin{table}[H]
    \small
    \centering
    \begin{tabular}{lccc|ccc}
        \toprule
                     & \multicolumn{3}{c}{\textbf{Two groups}} & \multicolumn{3}{c}{\textbf{More than two groups}}                                                                                                                                                                       \\
        \midrule
                     & \multicolumn{1}{c}{\textbf{Sex}}        & \multicolumn{1}{c}{\textbf{Student}}              & \multicolumn{1}{c}{\textbf{Continent}} & \multicolumn{1}{c}{\textbf{Nationality}} & \multicolumn{1}{c}{\textbf{Language}}  & \multicolumn{1}{c}{\textbf{Ethnicity}} \\
        \midrule
        \textbf{TP}  & 0.302                                   & \cellcolor[HTML]{EFEFEF}\textbf{0.032}            & 0.286                                  & 0.218                                    & 0.109                                  & 0.242                                  \\
        \textbf{TN}  & 0.726                                   & 0.379                                             & 0.204                                  & 0.190                                    & 0.216                                  & 0.281                                  \\
        \textbf{FP}  & 0.699                                   & 0.933                                             & 0.073                                  & \cellcolor[HTML]{EFEFEF}\textbf{0.020}   & \cellcolor[HTML]{EFEFEF}\textbf{0.040} & \cellcolor[HTML]{EFEFEF}\textbf{0.037} \\
        \textbf{FN}  & 0.961                                   & 0.150                                             & 0.847                                  & 0.478                                    & 0.438                                  & 0.584                                  \\
        \textbf{REJ} & 0.835                                   & 0.625                                             & 0.496                                  & 0.271                                    & 0.103                                  & 0.068                                  \\
        \bottomrule
    \end{tabular}
    \caption{\textbf{Aggregated}: an overview of the statistical differences between different groups of participants for various demographic characteristics for each aggregated scenario type in the ME survey. Each cell contains the p value of either the Mann-Whitney U test for two groups or the Kruskal-Wallis test for more than two groups. The grey cells with bold text indicate significant statistical differences between the groups for that feature and scenario type.}
    \label{tab:results-differences-grp}
\end{table}

\begin{table}[H]
    \scriptsize
    \centering
    \setlength\tabcolsep{2pt}
    \begin{subtable}{.44\textwidth}
        \centering
        \begin{tabular}{lcccc}
            \toprule
                                  & \textbf{South Africa} & \textbf{Poland}                        & \textbf{Portugal} & \textbf{Spain} \\
            \midrule
            \textbf{South Africa} & 1.000                 &                                        &                   &                \\
            \textbf{Poland}       & 0.077                 & 1.000                                  &                   &                \\
            \textbf{Portugal}     & 0.077                 & \cellcolor[HTML]{EFEFEF}\textbf{0.009} & 1.000             &                \\
            \textbf{Spain}        & 0.119                 & 0.083                                  & 0.613             & 1.000          \\
            \bottomrule
        \end{tabular}
        \caption{TP6}
    \end{subtable}%
    \begin{subtable}{.44\textwidth}
        \centering
        \begin{tabular}{cccc}
            \toprule
            \textbf{South Africa}                  & \textbf{Poland}                        & \textbf{Portugal}                      & \textbf{Spain} \\
            \midrule
            1.000                                  &                                        &                                        &                \\
            0.755                                  & 1.000                                  &                                        &                \\
            0.261                                  & 0.261                                  & 1.000                                  &                \\
            \cellcolor[HTML]{EFEFEF}\textbf{0.026} & \cellcolor[HTML]{EFEFEF}\textbf{0.038} & \cellcolor[HTML]{EFEFEF}\textbf{0.050} & 1.000          \\
            \bottomrule
        \end{tabular}
        \caption{FP2}
    \end{subtable}
    \vskip\baselineskip
    \begin{subtable}{.44\textwidth}
        \centering
        \begin{tabular}{lcccc}
            \toprule
                                  & \textbf{South Africa}                  & \textbf{Poland} & \textbf{Portugal} & \textbf{Spain} \\
            \midrule
            \textbf{South Africa} & 1.000                                  &                 &                   &                \\
            \textbf{Poland}       & 0.342                                  & 1.000           &                   &                \\
            \textbf{Portugal}     & 0.304                                  & 1.000           & 1.000             &                \\
            \textbf{Spain}        & \cellcolor[HTML]{EFEFEF}\textbf{0.043} & 0.304           & 0.220             & 1.000          \\
            \bottomrule
        \end{tabular}
        \caption{FP7}
    \end{subtable}%
    \begin{subtable}{.44\textwidth}
        \centering
        \begin{tabular}{cccc}
            \toprule
            \textbf{South Africa}                  & \textbf{Poland}                        & \textbf{Portugal} & \textbf{Spain} \\
            \midrule
            1.000                                  &                                        &                   &                \\
            \cellcolor[HTML]{EFEFEF}\textbf{0.034} & 1.000                                  &                   &                \\
            0.150                                  & \cellcolor[HTML]{EFEFEF}\textbf{0.011} & 1.000             &                \\
            \cellcolor[HTML]{EFEFEF}\textbf{0.045} & \cellcolor[HTML]{EFEFEF}\textbf{0.011} & 0.679             & 1.000          \\
            \bottomrule
        \end{tabular}
        \caption{FN5}
    \end{subtable}
    \vskip\baselineskip
    \begin{subtable}{.44\textwidth}
        \centering
        \begin{tabular}{lcccc}
            \toprule
                                  & \textbf{South Africa} & \textbf{Poland} & \textbf{Portugal} & \textbf{Spain} \\
            \midrule
            \textbf{South Africa} & 1.000                 &                 &                   &                \\
            \textbf{Poland}       & 0.095                 & 1.000           &                   &                \\
            \textbf{Portugal}     & 0.622                 & 0.095           & 1.000             &                \\
            \textbf{Spain}        & 0.104                 & 0.095           & 0.104             & 1.000          \\
            \bottomrule
        \end{tabular}
        \caption{FN6}
    \end{subtable} %
    \begin{subtable}{.44\textwidth}
        \centering
        \begin{tabular}{cccc}
            \toprule
            \textbf{South Africa} & \textbf{Poland} & \textbf{Portugal} & \textbf{Spain} \\
            \midrule
            1.000                 &                 &                   &                \\
            0.088                 & 1.000           &                   &                \\
            0.227                 & 0.088           & 1.000             &                \\
            1.000                 & 0.227           & 0.388             & 1.000          \\
            \bottomrule
        \end{tabular}
        \caption{REJ1}
    \end{subtable}%
    \vskip\baselineskip
    \begin{subtable}{.44\textwidth}
        \centering
        \begin{tabular}{lcccc}
            \toprule
                                  & \textbf{South Africa} & \textbf{Poland} & \textbf{Portugal} & \textbf{Spanish} \\
            \midrule
            \textbf{South Africa} & 1.000                 &                 &                   &                  \\
            \textbf{Poland}       & 0.098                 & 1.000           &                   &                  \\
            \textbf{Portugal}     & 0.098                 & 1.000           & 1.000             &                  \\
            \textbf{Spain}        & 0.098                 & 0.422           & 0.422             & 1.000            \\
            \bottomrule
        \end{tabular}
        \caption{REJ4}
    \end{subtable}
    \caption{\textbf{Nationality}: an overview of all pairwise Mann-Whitney U tests between the different nationalities for all scenarios where we found significant differences between all nationalities using the Kruskal-Wallis test. Each cell contains the p value of the Mann-Whitney U test between two groups of different nationalities. We corrected all p values with the Benjamini-Hochberg procedure. The grey cells with bold text indicate significant statistical differences between the two nationalities.}
    \label{tab:results-pairwise-nationality}
\end{table}

\begin{table}[H]
    \scriptsize
    \centering
    \setlength\tabcolsep{2pt}
    \begin{subtable}{.405\textwidth}
        \centering
        \begin{tabular}{lcccc}
            \toprule
                               & \textbf{English} & \textbf{Polish} & \textbf{Portugese} & \textbf{Spanish} \\
            \midrule
            \textbf{English}   & 1.000            &                 &                    &                  \\
            \textbf{Polish}    & 0.561            & 1.000           &                    &                  \\
            \textbf{Portugese} & 0.352            & 0.283           & 1.000              &                  \\
            \textbf{Spanish}   & 0.176            & 0.258           & 0.142              & 1.000            \\
            \bottomrule
        \end{tabular}
        \caption{TP3}
    \end{subtable}%
    \begin{subtable}{.405\textwidth}
        \centering
        \begin{tabular}{cccc}
            \toprule
            \multicolumn{1}{c}{\textbf{English}} & \multicolumn{1}{c}{\textbf{Polish}}    & \textbf{Portugese} & \textbf{Spanish} \\
            \midrule
            1.000                                & \multicolumn{1}{c}{}                   &                    &                  \\
            0.089                                & 1.000                                  &                    &                  \\
            0.119                                & \cellcolor[HTML]{EFEFEF}\textbf{0.007} & 1.000              &                  \\
            0.522                                & 0.089                                  & 1.000              & 1.000            \\
            \bottomrule
        \end{tabular}
        \caption{TP6}
    \end{subtable}
    \vskip\baselineskip
    \begin{subtable}{.405\textwidth}
        \centering
        \begin{tabular}{lcccc}
            \toprule
                               & \textbf{English}                       & \textbf{Polish} & \textbf{Portugese} & \textbf{Spanish} \\
            \midrule
            \textbf{English}   & 1.000                                  &                 &                    &                  \\
            \textbf{Polish}    & 0.321                                  & 1.000           &                    &                  \\
            \textbf{Portugese} & 0.321                                  & 1.000           & 1.000              &                  \\
            \textbf{Spanish}   & \cellcolor[HTML]{EFEFEF}\textbf{0.019} & 0.444           & 0.321              & 1.000            \\
            \bottomrule
        \end{tabular}
        \caption{FP7}
    \end{subtable}%
    \begin{subtable}{.405\textwidth}
        \centering
        \begin{tabular}{cccc}
            \toprule
            \textbf{English} & \textbf{Polish}                        & \textbf{Portugese} & \textbf{Spanish} \\
            \midrule
            1.000            &                                        &                    &                  \\
            0.070            & 1.000                                  &                    &                  \\
            0.209            & \cellcolor[HTML]{EFEFEF}\textbf{0.011} & 1.000              &                  \\
            0.647            & 0.164                                  & 0.838              & 1.000            \\
            \bottomrule
        \end{tabular}
        \caption{FN5}
    \end{subtable}
    \vskip\baselineskip
    \begin{subtable}{.405\textwidth}
        \centering
        \begin{tabular}{lcccc}
            \toprule
                               & \textbf{English}                       & \textbf{Polish} & \textbf{Portugese} & \textbf{Spanish} \\
            \midrule
            \textbf{English}   & 1.000                                  &                 &                    &                  \\
            \textbf{Polish}    & 0.895                                  & 1.000           &                    &                  \\
            \textbf{Portugese} & 0.439                                  & 0.721           & 1.000              &                  \\
            \textbf{Spanish}   & \cellcolor[HTML]{EFEFEF}\textbf{0.049} & 0.309           & 0.548              & 1.000            \\
            \bottomrule
        \end{tabular}
        \caption{FN7}
    \end{subtable}%
    \begin{subtable}{.405\textwidth}
        \centering
        \begin{tabular}{cccc}
            \toprule
            \textbf{English} & \textbf{Polish} & \textbf{Portugese} & \textbf{Spanish} \\
            \midrule
            1.000            &                 &                    &                  \\
            0.076            & 1.000           &                    &                  \\
            0.387            & 0.076           & 1.000              &                  \\
            0.149            & 0.711           & 0.096              & 1.000            \\
            \bottomrule
        \end{tabular}
        \caption{REJ1}
    \end{subtable}
    \vskip\baselineskip
    \begin{subtable}{.405\textwidth}
        \centering
        \begin{tabular}{lcccc}
            \toprule
                               & \textbf{English} & \textbf{Polish} & \textbf{Portugese} & \textbf{Spanish} \\
            \midrule
            \textbf{English}   & 1.000            &                 &                    &                  \\
            \textbf{Polish}    & 0.063            & 1.000           &                    &                  \\
            \textbf{Portugese} & 0.063            & 1.000           & 1.000              &                  \\
            \textbf{Spanish}   & 0.599            & 1.000           & 1.000              & 1.000            \\
            \bottomrule
        \end{tabular}
        \caption{REJ4}
    \end{subtable}%
    \begin{subtable}{.405\textwidth}
        \centering
        \begin{tabular}{cccc}
            \toprule
            \textbf{English} & \textbf{Polish} & \textbf{Portugese} & \textbf{Spanish} \\
            \midrule
            1.000            &                 &                    &                  \\
            1.000            & 1.000           &                    &                  \\
            0.489            & 0.452           & 1.000              &                  \\
            0.105            & 0.152           & 0.105              & 1.000            \\
            \bottomrule
        \end{tabular}
        \caption{REJ5}
    \end{subtable}
    \caption{\textbf{Language}: an overview of all pairwise Mann-Whitney U tests between the different spoken languages for all scenarios where we found significant differences between all spoken languages using the Kruskal-Wallis test. Each cell contains the p value of the Mann-Whitney U test between two groups of languages. We corrected all p values with the Benjamini-Hochberg procedure. The grey cells with bold text indicate significant statistical differences between the two languages.}
    \label{tab:results-pairwise-language}
\end{table}

\begin{table}[H]
    \scriptsize
    \centering
    \setlength\tabcolsep{2pt}
    \begin{subtable}{.275\textwidth}
        \centering
        \begin{tabular}{lccc}
            \toprule
                           & \textbf{White}                         & \textbf{Mixed}                         & \textbf{Black} \\
            \midrule
            \textbf{White} & 1.000                                  &                                        &                \\
            \textbf{Mixed} & 0.552                                  & 1.000                                  &                \\
            \textbf{Black} & \cellcolor[HTML]{EFEFEF}\textbf{0.028} & \cellcolor[HTML]{EFEFEF}\textbf{0.028} & 1.000          \\
            \bottomrule
        \end{tabular}
        \caption{FP7}
    \end{subtable}%
    \begin{subtable}{.21\textwidth}
        \centering
        \begin{tabular}{ccc}
            \toprule
            \multicolumn{1}{l}{\textbf{White}}     & \textbf{Mixed}                                             & \textbf{Black}            \\
            \midrule
            1.000                                  &                                                            &                           \\
            0.776                                  & \multicolumn{1}{r}{1.000}                                  &                           \\
            \cellcolor[HTML]{EFEFEF}\textbf{0.002} & \multicolumn{1}{r}{\cellcolor[HTML]{EFEFEF}\textbf{0.776}} & \multicolumn{1}{r}{1.000} \\
            \bottomrule
        \end{tabular}
        \caption{REJ4}
    \end{subtable}%
    \begin{subtable}{.21\textwidth}
        \centering
        \begin{tabular}{ccc}
            \toprule
            \multicolumn{1}{l}{\textbf{White}}     & \textbf{Mixed}            & \textbf{Black}            \\
            \midrule
            1.000                                  &                           &                           \\
            \cellcolor[HTML]{EFEFEF}\textbf{0.016} & \multicolumn{1}{r}{1.000} &                           \\
            1.000                                  & \multicolumn{1}{r}{0.112} & \multicolumn{1}{r}{1.000} \\
            \bottomrule
        \end{tabular}
        \caption{REJ7}
    \end{subtable}
    \caption{\textbf{Ethnicity}: an overview of all pairwise Mann-Whitney U tests between the different ethnicities for all scenarios where we found significant differences between all ethnicities using the Kruskal-Wallis test. Each cell contains the p value of the Mann-Whitney U test between two groups of ethnicities. We corrected all p values with the Benjamini-Hochberg procedure. The grey cells with bold text indicate significant statistical differences between the two ethnicities.}
    \label{tab:results-pairwise-ethnicity}
\end{table}

\section{Probability Density Functions}
\begin{figure}[H]
    \centering
    \begin{subfigure}{.45\textwidth}
        \includegraphics[scale=.35]{Figures/confidence-densities-seen-tp.pdf}
        \caption{TP}
    \end{subfigure}
    \begin{subfigure}{.45\textwidth}
        \includegraphics[scale=.35]{Figures/confidence-densities-seen-tn.pdf}
        \caption{TN}
    \end{subfigure}
    \begin{subfigure}{.45\textwidth}
        \includegraphics[scale=.35]{Figures/confidence-densities-seen-fp.pdf}
        \caption{FP}
    \end{subfigure}
    \begin{subfigure}{.45\textwidth}
        \includegraphics[scale=.35]{Figures/confidence-densities-seen-fn.pdf}
        \caption{FN}
    \end{subfigure}
    \caption{\textbf{Seen data}: Probability Density Functions of the confidence values of all predictions for the \emph{seen data} estimated with Kernel Density Estimation.}
    \label{fig:pdfs-seen}
\end{figure}

\begin{figure}[H]
    \centering
    \begin{subfigure}{.45\textwidth}
        \includegraphics[scale=.35]{Figures/confidence-densities-unseen-tp.pdf}
        \caption{TP}
    \end{subfigure}
    \begin{subfigure}{.45\textwidth}
        \includegraphics[scale=.35]{Figures/confidence-densities-unseen-tn.pdf}
        \caption{TN}
    \end{subfigure}
    \begin{subfigure}{.45\textwidth}
        \includegraphics[scale=.35]{Figures/confidence-densities-unseen-fp.pdf}
        \caption{FP}
    \end{subfigure}
    \begin{subfigure}{.45\textwidth}
        \includegraphics[scale=.35]{Figures/confidence-densities-unseen-fn.pdf}
        \caption{FN}
    \end{subfigure}
    \caption{\textbf{Unseen data}: Probability Density Functions of the confidence values of all predictions for the \emph{unseen data} estimated with Kernel Density Estimation.}
    \label{fig:pdfs-unseen}
\end{figure}

%----------------------------------------------------------------------------------------
%	BIBLIOGRAPHY
%----------------------------------------------------------------------------------------

\printbibliography[heading=bibintoc]

%----------------------------------------------------------------------------------------

\end{document}
