\chapter{Survey study}
\label{sec:survey}
The second part of this research is to find out how we can determine the value ratios between TP, TN, FP, FN, and rejected predictions.
%
We conducted a literature study in section \ref{sec:related-work-value-assessment} and concluded that we want to emperically estimate the value ratios from the perspective of the social media user.
%
In section \ref{sec:me}, we found that Magnitude Estimation (ME) seems like a promising technique for estimating these value ratios.
%
Therefore, in this chapter, we discuss how we apply the ME technique in a crowdsourced survey study.
%

%
We design a survey study to ask participants the degree to which they agree or disagree with the decisions of a fictional social media platform called SocialNet.
%
We show different scenarios to the participants that represent TP, TN, FP, FN, and rejected predictions.
%
The TP and TN scenarios mean that SocialNet successfully detects whether a post is hateful or not, respectively.
%
The FP scenario means that SocialNet incorrectly predicts a non-hateful post as hateful, and conversely for the FN scenario.
%
For example, in the FN scenario, the survey shows a hateful post to the subject and explains that SocialNet did not identify the post as hate speech.
%
Then, participants indicate the degree of agreement/disagreement using some scales, and the answers per scenario are aggregated to obtain the relative values.
%

%
The structure and preparation of our crowdsourced survey study follows the pre-registration plan for social psychology suggested by \citet{van2016pre}.
%
In a pre-registration plan, we describe the hypothesis, procedure, and analysis before conducting the crowdsourced survey study to increase scientific credibility, reproducibility, and to reduce bias \citep{van2016pre}.
%
It is important to select the statistical methods for the analysis part beforehand to prevent ourselves from selecting the statistic that best fits the collected data.
%
The content of this chapter is based on the final version of the pre-registration plan created after conducting the pilot survey.
%

%
In section \ref{sec:survey-hypothesis}, we make an hypothesis about the ME method and the value ratios.
%
Section \ref{sec:survey-method} contains all details about the setup of the survey.
%
Finally, in section \ref{sec:survey-analysis}, we elaborate on the analysis of the survey results.

\section{Hypothesis}
\label{sec:survey-hypothesis}
% "Describe the (numbered) hypotheses in terms of relationships between your variables. 2. For interaction effects, describe the expected shape of the interactions. 3. If you are manipulating a variable, make predictions for successful check variables or explain why no manipulation check is included. 4. A figure or table may be helpful to describe complex interactions. 5. For original research, add rationales or theoretical frameworks for why a certain hypothesis is tested. 6. If multiple predictions can be made for the same IV-DV combination, describe what outcome would be predicted by which theory"

\section{Method}
\label{sec:survey-method}
\subsection{Scales}

\subsection{Normalization}

\subsection{Design}
\subsubsection{Independent variables}
\subsubsection{Confounding variables}
\subsubsection{Control variables}
\subsubsection{Dependent variables}

\subsection{Planned sample}
\subsubsection{Sample size}

\subsubsection{Subjects}
\todo[inline]{Explain subject Inclusion and Exclusion Criteria}
\todo[inline]{Explain subject Compensation}

\subsection{Materials}
\subsubsection{Survey tool}
\subsubsection{Data}

\subsection{Procedure}

\section{Analysis}
\label{sec:survey-analysis}

\subsection{Validation}
\subsection{Reliability}


