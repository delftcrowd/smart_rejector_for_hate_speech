\chapter{Survey study}
\label{sec:survey}
The second part of this research is to find out how we can determine the value ratios between TP, TN, FP, FN, and rejected predictions.
%
We conducted a literature study in section \ref{sec:related-work-value-assessment} and concluded that we want to emperically estimate the value ratios from the perspective of the social media user.
%
In section \ref{sec:me}, we found that Magnitude Estimation (ME) seems like a promising technique for estimating these value ratios.
%
Therefore, in this chapter, we discuss how we apply the ME technique in a crowdsourced survey study.
%

%
We design a survey study to ask participants the degree to which they agree or disagree with the decisions of a fictional social media platform called SocialNet.
%
We show different scenarios to the participants that represent TP, TN, FP, FN, and rejected predictions.
%
The TP and TN scenarios mean that SocialNet successfully detects whether a post is hateful or not, respectively.
%
The FP scenario means that SocialNet incorrectly predicts a non-hateful post as hateful, and conversely for the FN scenario.
%
For example, in the FN scenario, the survey shows a hateful post to the participant and explains that SocialNet did not identify the post as hate speech.
%
Then, participants indicate the degree of agreement/disagreement using some scales, and the answers per scenario are aggregated to obtain the relative values.
%

%
The structure and preparation of our crowdsourced survey study follows the pre-registration plan for social psychology suggested by \citet{van2016pre}.
%
In a pre-registration plan, we describe the hypothesis, procedure, and analysis before conducting the crowdsourced survey study to increase scientific credibility, reproducibility, and to reduce bias \citep{van2016pre}.
%
It is important to select the statistical methods for the analysis part beforehand to prevent ourselves from selecting the statistic that best fits the collected data.
%
The content of this chapter is based on the final version of the pre-registration plan created after conducting the pilot survey.
%

%
In section \ref{sec:survey-hypothesis}, we make an hypothesis about the ME method and the value ratios.
%
Section \ref{sec:survey-method} contains all details about the setup of the survey.
%
Finally, in section \ref{sec:survey-analysis}, we elaborate on the analysis of the survey results.

\section{Hypothesis}
\label{sec:survey-hypothesis}
Before we conducted the survey experiment, we listed several hypotheses about the value ratios and the ME method:
\begin{itemize}
    \item \textbf{We hypothesize that the values of FP and FN are negative and that the value of an FN is lower than an FP}. We believe that both FP and FN predictions have a negative impact on the social media users and, therefore, we think that both values should be negative. We believe that allowing hateful content to be publicly visible has a more negative impact on social media users than filtering out neutral content. Therefore, we think that the value of an FN is smaller than the value of an FP.
    \item \textbf{We hypothesize that the values of TP and TN are both positive and that the value of an TP is greater than of an TN.} We believe that both TP and TN predictions have a positive impact on the social media users and, therefore, we think that both values should be positive. We believe that correctly predicting hateful content is more valuable to social media users than correctly predicting non-hateful content. Therefore, we think that the value of an TP is greater than the value of an TN.
    \item \textbf{We hypothesize that the value of a rejection is negative and that it is greater than the average value of an FP and an FN.} The key assumption of using ML models with a reject option is that the negative value of a rejection should always be greater than the negative value of an incorrect decision.
    \item \textbf{We hypothesize that Magnitude Estimation (ME) is a suitable technique for retrieving the value ratios.} ME seems like a promising technique for retrieving ratio data from judgements about hate speech detection scenarios. We use a 100-level numerical scale for validation. We expect that both scales are correlated and will give similar judgements. Although we also expect the 100-level scale to be suitable for retrieving opinions about the different hate speech detection scenarios, it does not provide the ratio data we need. We also expect that the inter-rater reliability for the 100-level scale will be higher than for the ME scale since the ME scale provides more response freedom. We also expect this since the authors of \cite{roitero2018fine} concluded that the inter-rater reliability of the 100-level scale is higher than the ME scale when rating the relevance of documents.
\end{itemize}

\section{Method}
\label{sec:survey-method}
In this section, we discuss the complete setup of the survey experiment and how we use both scales.

\subsection{Scales}
We use ME as the primary scale of our survey experiment.
%
Like we concluded in the section \ref{sec:me}, we also need to validate the the ME scale.
%
We validate the ME scale through cross-modality validation by comparing the results of the ME scale with another scale as explained in section \ref{sec:me}.
%
The secondary scale is a bounded scale that consists of 100 levels, called the 100-level scale  for four reasons.
%
First, it is impractical in this project, given the limited budget, to use other ME scales, such as measuring the intensity of the participants' handgrips to express their judgements.
%
Second, there does not exist any suitable dataset that we can use for validation that contains human ratings of different scenarios in hate speech detection.
%
Third, we concluded in \ref{sec:likert} that Likert scales have limited response freedom.
%
Finally, in \cite{roitero2018fine}, the authors concluded that the 100-level scale provides more response freedom than course-grained Likert scales and has several advantages over ME in terms of usability and reliability.
%
The 100-level scale is easier to understand than ME, does not require normalization, and provides more flexibility than Likert scales \cite{roitero2018fine}.
%
Therefore, we will create two separate surveys with the same scenarios where half of all participants use the 100-level scale and the other half use the ME scale.
%
Both scales are bipolar scales since the participants should be able to either disagree or agree with the scenarios.

\subsection{Normalization}
The ME scale is unbounded and, therefore, provides a lot of response freedom.
%
For example, suppose we first show a scenario and the participant provides a value (e.g., 100) to indicate the degree of agreement.
%
Suppose we next present a scenario that the participant agrees with more.
%
The participant can always provide a higher value (e.g., 125) and not be restricted within a fixed range.
%
However, the results need to be normalized as different participants rate the agreement/disagreement degree differently.
%
As explained in section \ref{sec:me}, we cannot use common normalization methods such as geometric averaging as we are using bipolar scales with negative values.
%
Therefore, we normalize the results by dividing the magnitude estimates of each participant by their maximum estimate.
%
We multiply the normalized magnitude estimates by 100 for the sake of clarity.
%
This way, all magnitudes estimates are in the range $[-100, 100]$ while maintaining the ratio properties.

\subsection{Design}
We will list all independent, dependent, confounding, and control variables analyzed in our experiment in this section.

\subsubsection{Independent variables}
Independent variables are the different hate speech detection scenarios we show to the participants (TP, TN, FP, FN, and rejection).
%
We inform the participants in the case of TP and FP scenarios that SocialNet ranks the hateful post lower on their feed.
%
The users then need to spend more effort finding the post since they need to scroll longer before it becomes visible.
%
We decided to explain that the hateful posts are ranked lower since we found significant disagreement among participants after the pilot survey in which we explained that SocialNet removes hateful posts.
%
We inform participants in the rejection scenarios that a human moderator needs to check the post (that can be either hateful or not hateful) within 24 hours.
%
Meanwhile, the post remains visible with its original rank on the user's feed.
%
We use 24 hours based on the German NetzDG law, which allows the government to fine social media platforms if they do not remove illegal hate speech within 24 hours \citep{tworek2019analysis}.
%
\begin{itemize}
    \item \textbf{True Positive} Show a hateful post to the user and explain that SocialNet detected hate and ranked the post lower on people's feeds.
    \item \textbf{True Negative} Show a non-hateful post to the user and explain that SocialNet did not detect hate and allowed the post.
    \item \textbf{False Positive} Show a non-hateful post to the user and explain that SocialNet detected hate and ranked the post lower on people's feeds.
    \item \textbf{False Negative} Show a hateful post to the users and explain that SocialNet did not detect hate and allowed the post.
    \item \textbf{Rejection}
          \begin{itemize}
              \item Show a hateful post to the user and explain that SocialNet was uncertain whether the post was hateful or not. An internal moderator will need to check the post within 24 hours. Meanwhile the post remains visible.
              \item Or, show a non-hateful post to the user and explain that SocialNet was uncertain whether the post was hateful or not. An internal moderator will need to check the post within 24 hours. Meanwhile the post remains visible.
          \end{itemize}
\end{itemize}

\subsubsection{Confounding variables}
%
Confounding variables are demographics, such as gender, education, location, age, etc.
%
\begin{itemize}
    \item \textbf{Nationality} People from different nationalities might have different perceptions and definitions of hate speech and how we should deal with it (tolerating hateful content or removing it).
    \item \textbf{Age} People of different ages might have different perceptions and definitions of hate speech and how we should deal with it.
    \item \textbf{Educational level} People with different educational levels might have different perceptions and definitions of hate speech and how we should deal with it.
    \item \textbf{Gender} According to \cite{gold2018women}, there is no significant difference in how men and women perceive hate. However, we still report gender as a confounding variable since we want to analyze if there are genuinely not any differences.
\end{itemize}

\subsubsection{Control variables}
%
We define two control variables: the measurement scales and the content of the social media posts we show to the participants.
%
We control the measurement scale variable by randomly assigning a participant to use either the 100-level or the ME scale to rate the scenarios.
%
Regarding the scales, as described before, we choose Magnitude Estimation as our primary scale and use the 100-level scale for validation.
%
We leave the study of other scales to future work.
%
We control the content of the social media posts in two manners.
%
First, we present all scenarios for all participants randomly to reduce bias.
%
Second, we sample the social media posts for the survey from existing datasets.
% 
We explain the selection procedure in section \ref{sec:data}.
%
\begin{itemize}
    \item \textbf{Scales} The first group of participants must answer the questions using the ME scale. The second group needs to answer the questions using the 100-level scale.
    \item \textbf{Content of the posts} We sample all social media posts from existing datasets and present them to the participants in random order.
\end{itemize}

\subsubsection{Dependent variables}
Our dependent variables are reliability, validity and the value ratio of TP, TN, FP, FN, and rejection scenarios.
%
\begin{itemize}
    \item \textbf{Reliability} Measured using Krippendorff's alpha, where values larger than 0.8 indicate reliable conclusions and values larger than 0.6 indicate tentative conclusions \cite{krippendorff2004reliability}.
    \item \textbf{Validity} Convergent validity, if two different measures measure the same thing \cite{fitzner2007reliability}. Measured by calculating the correlation between the magnitude estimates and the response values from the 100-level scale.
    \item \textbf{Value ratios of TP, TN, FP, FN, and rejection scenarios} Measured by calculating the median of the normalized magnitude estimates of each scenario question and then calculating the mean over the resulting values to come up with the final value for that scenario type.
\end{itemize}

\subsection{Planned sample}
This section discusses how we pick the sample size, recruit the participants, and explains which stopping and exclusion rules we apply.

\subsubsection{Sample size}
There are 4.55 billion active social media users\footnote{\url{https://datareportal.com/reports/digital-2021-october-global-statshot}}.
%
We choose a 90\% Confidence Interval (CI) and 10\% Margin of Error (MoE) for this study.
%
So for 90\% of the time, our observations will fall within a 10\% interval \cite{olson2014ways}.
%
According to \cite{olson2014ways}, we need a sample size of 68 participants per survey type to reach the desired CI and MoE values.
%
We choose 10\% MoE since we have a limiting budget.
%
We first conduct a pilot survey for 12 participants per scale to gather feedback and check if we need to improve things before the actual experiment.
%
We want to determine the average workload using the pilot survey and decide whether it's possible to reduce the MoE by increasing the number of participants.
%
For the pilot survey, we use 24 participants.
%
Therefore, in total we will need $2*12+2*68 = 160$ participants.
%
Of the recruited participants, 50\% identified as female.
%
Half of the participants are assigned the ME scale and the other half the 100-level scale.

\subsubsection{Participants}
We will use the \href{https://prolific.co}{Prolific} platform for recruiting online participants for the survey study.
%
We will use the following inclusion criteria for our participants:
\begin{itemize}
    \item 18 years of age and older since we are showing offensive language in the experiment.
    \item Fluent in English.
    \item Approval rating over 90\% on the Prolific platform.
    \item Use one of the following social media platforms regularly (at least once a month): Facebook, Twitter, YouTube, LinkedIn, Pinterest, Google Plus, Tumblr, Instagram, Reddit, VK, Flickr, Vine.co, Meetup, ask.fm, Snapchat, TikTok, Medium.
\end{itemize}
%
Every participant will be paid based on the hourly wage of 9.0 GBP (about 10,67 Euro), indicated as good pay by the platform\footnote{\url{https://prolific.co/pricing}}.
%
We use the following exclusion/rejection criteria:
\begin{itemize}
    \item Participants who fail the two attention checks. We will include two Instructional Manipulation Checks to check if the user pays attention to the survey\footnote{\url{https://researcher-help.prolific.co/hc/en-gb/articles/360009223553}}.
    \item Participants who do not complete all questions.
    \item Participants who do not agree with the informed consent before the start of the survey. We are not allowed to collect and process their data if they do not consent.
    \item Participants who do fail the ME training phase, for example by providing random values that do not make any sense.
\end{itemize}

\subsection{Data}
\label{sec:data}
We use 40 different social media posts where 20 are hateful and 20 are not hateful to create 40 different scenarios.
%
All subjects judge four scenarios of each type (TP, TN, FP, FN, and rejection) for each scale type (ME and 100-level) since we let each subject use both scales.
%
We select the posts from a public dataset \citep{basile2019semeval} that contains 13,000 English tweets.
%
Each tweet is annotated with three categories: hate speech (yes/no), target (generic group or an individual), and aggressiveness (yes/no).
%
Therefore, we have four different groups of hateful tweets: generic target + aggressive, individual target + aggressive, generic target + non-aggressive, and individual target + aggressive.
%
We exclude all tweets that contain replies and mentions since  they have unclear contexts.
%
Then we preprocess all tweets by removing the URLs and hashtags.
%
Finally, we use clustering analysis to select 40 tweets in our study.
%
We perform term frequency-inverse document frequency (TF-IDF), latent semantic analysis (LSA), and k-means clustering on each group of tweets.
%
We calculate the silhouette coefficient to determine the optimal cluster size (k value) for the neutral tweets and the four groups of hateful tweets.
%
The silhouette analysis indicates to set k as large as possible.
%
We select the five nearest data samples to each cluster centroid.
%
From this selection, we manually select one tweet per cluster using a majority vote from three members from our group to create the final set of 40 tweets.
%
Based on the silhouette coefficient, we use a cluster size of 20 for the neutral tweets and select one tweet per cluster to collect 20 neutral tweets.
%
And we use a cluster size of 5 for each group of hateful tweets to collect 20 hateful tweets.

\subsection{Procedure}

\section{Analysis}
\label{sec:survey-analysis}

\subsection{Validation}
\subsection{Reliability}


