\chapter{Results}
This chapter includes the results of the survey study (chapter \ref{ch:survey}) and the experiments with the value-sensitive rejector (chapter \ref{ch:rejector}).
%
We first present the results of the survey study as the experiments with the value-sensitive rejector depended on the outcomes of the survey study.
%

%
The goal of the survey study was to retrieve the value ratios of TP, TN, FP, FN, and rejected predictions in hate speech detection from the user's perspective.
%
We retrieved the value ratios using a scale called Magnitude Estimation (ME) and validated the ME scale by conducting a separate survey that uses a bounded scale that consists of 100 levels, called the 100-level scale.
%
The goals of the experiments with the value-sensitive rejector were finding out how the rejector behaves on different models and datasets, if ML with a reject option is beneficial for hate speech detection, and comparing the value-sensitive metric against machine metrics such as accuracy.
%

%
In section \ref{sec:results-survey-study}, we present the results of the complete  survey study that we gathered after conducting the pilot survey, and in section \ref{sec:results-rejector}, we present the results of the experiments with the value-sensitive rejector.

\section{Survey study}
\label{sec:results-survey-study}
This section presents the value ratios in section \ref{sec:results-value-ratios}, the reliability analysis in section \ref{sec:results-reliability}, the validity analysis in section \ref{sec:results-validity}, and the demographic analysis in section \ref{sec:results-demographics}.
%


\subsection{Value ratios}
\label{sec:results-value-ratios}
As explained in section \ref{sec:analysis-values}, we first calculated the median of all responses for each question.
%
We use the median since we found in the pilot survey that the data from both scales are highly skewed with outliers.
%
Then, we calculate the mean of all values with the same scenario type (TP, TN, FP, FN, or rejection).
%
We show the resulting value ratios in table \ref{tab:values-reliability}.
%
We

\begin{table}[t]
    \centering
    \begin{tabular}{lcccc}
        \toprule
                           & \multicolumn{2}{c}{\textbf{ME}} & \multicolumn{2}{c}{\textbf{100-level}}                                        \\
        \cmidrule(l){2-3} \cmidrule(l){4-5}
                           & $\boldsymbol{\alpha}$           & $\textbf{v}$                           & $\boldsymbol{\alpha}$ & $\textbf{v}$ \\
        \midrule
        \textbf{TP}        & 0.07                            & 18.15                                  & 0.04                  & 77.00        \\
        \textbf{TN}        & 0.10                            & 36.32                                  & 0.11                  & 86.31        \\
        \textbf{FP}        & 0.39                            & -16.69                                 & 0.07                  & -51.00       \\
        \textbf{FN}        & 0.92                            & -28.08                                 & 0.14                  & -62.43       \\
        \textbf{Rejection} & -0.31                           & -4.82                                  & 0.07                  & -16.37       \\
        \midrule
        \textbf{All}       & 0.78                            & ---                                    & 0.44                  & ---          \\
        \bottomrule
    \end{tabular}
    \caption{Krippendorff's alpha ($\alpha$) and the scenario values ($v$) for TP, TN, FP, FN, and rejection scenarios for both scales: the Magnitude Estimation (ME) and 100-level scale.}
    \label{tab:values-reliability}
\end{table}

\subsection{Reliability}
\label{sec:results-reliability}

\subsection{Validity}
\label{sec:results-validity}

\subsection{Demographics}
\label{sec:results-demographics}

\section{Value-sensitive rejection}
\label{sec:results-rejector}
\todo[inline]{Explain that the metric conditions still hold if we set $V_{tp}$ and $V_{tn}$ to 0}
