\chapter{Introduction}
\label{ch:introduction}
\newcommand{\customtextbox}[1]{
	\setlength{\fboxsep}{0.5em}
	\fbox{
		\begin{minipage}{\linewidth-1.7em}
			\vspace*{0.25em}
			#1
		\end{minipage}
	}
}

The amount of hateful content spread online on social media remains a significant problem.
%
Ignoring its presence can harm people and even result in actual violence and other conflicts \citep{ecri-hate-speech-and-violence, balayn2021automatic}.
%
Many news articles exist about events where hate spread on online platforms leads to acts of violence \citep{columbia-facebook-linked-to-violence, mujib-mashal-india, paul-mozur-2018, muller2021fanning}.
%
One research paper found a connection between hateful content on Facebook containing anti-refugee sentiment and hate crimes against refugees by analyzing social media usage in multiple municipalities in Germany \citep{muller2021fanning}.
%
Governmental institutions and social media companies are becoming more aware of these risks and are trying to combat hate speech.
%
For example, the European Union developed a Code of Conduct for countering illegal hate speech in cooperation with large social media companies such as Facebook and Twitter \citep{eu-code-of-conduct}.
%
This Code of Conduct requests companies to prohibit hate speech and report yearly progress \citep{eu-code-of-conduct}.
%
The most recent report from 2021 stated that Twitter only removed 49.5\% of all hateful content on its platform.
%
Facebook is most successful in removing hate speech, claiming to have removed 70.2\% of all hateful content in 2021 \citep{eu-code-of-conduct}.
%
However, one article found in internal communication from Facebook that this percentage is much lower, around 3-5\% \citep{noah2021giansiracusa}.
%
Therefore, hate speech detection remains a complex problem that even large institutions have not solved yet.

\section{Problem statement}
Currently, people rely on reactive and proactive content moderation methods to detect hate speech \citep{klonick2017new}.
%
Reactive moderation is when social media users flag hateful content (also known as reporting) \citep{klonick2017new}.
%
Proactive moderation is done automatically using detection algorithms or manually by a group of human moderators \citep{klonick2017new}.
%
There exist different methods for automatically detecting hateful content.
%
Most use machine learning (ML) algorithms since these tend to be the most promising for their detection performance at a large scale \citep{balayn2021automatic, fortuna2018survey}.
%
These algorithms range from traditional ML methods, such as support vector machines (SVM) or decision trees, to deep learning (DL) algorithms \citep{fortuna2018survey}.
%

%
However, both proactive and reactive moderation methods have their limitations.
%
Proactive manual moderation of hateful content is still the most reliable solution, but it is simply infeasible because of the large amount of content generated by the many users \citep{balayn2021automatic}.
%
Reactive moderation solves this problem since the users can report hate speech themselves.
%
Although, the problem stays that users are still exposed to hateful content for some time.
%
Proactive automatic moderation using automated detection algorithms allows large amounts of data to be checked quickly without the involvement of humans.
%
However, these algorithms are unreliable as they often perform poorly on deployment data \citep{balayn2021automatic, grondahl2018all, arango2019hate}.
%
One study found that the F1 scores reduce significantly (69\% F1 score drop in the worst case) when training a hate speech detection model on one dataset and evaluating it using another \citep{grondahl2018all}.
%
Furthermore, one paper found that most research in hate speech detection overestimates the performance of the automated detection methods \citep{arango2019hate}.
%
Likewise, the authors noticed significant performance drops when the detection algorithms were trained on one dataset and evaluated on another \citep{arango2019hate}.
%

%
Therefore, humans and machines should work together to detect hate speech.
%
ML models should detect hateful content automatically, and humans should make the final decisions (\textit{human-in-the-loop}) when the model is not confident enough \citep{woo2020future, hendrickx2021machine}.
%
The challenge is to determine when we can accept ML predictions and when we need to reject them and defer them to a human moderator.
%
We focus on deciding whether to accept or reject ML predictions by taking the context-dependent \emph{values} into account.
%
Several papers advocate for integrating context-dependent values into the design of human-AI systems \citep{casati2021value,sayin2021science,zhu2018value, umbrello2021mapping, cummings2006integrating}.
%
There are benefits of correct predictions (positive value), costs of incorrect predictions (negative value), and costs of rejecting predictions.
%
More specifically, the values for false negative (FN) predictions, labelling something as non-hateful when it is, and false positive (FP) predictions, labelling something as hateful when it is not, might differ.
%
We should weigh these values according to the task of hate speech detection and incorporate them in the decision of accepting or rejecting ML predictions \citep{sayin2021science}.
%
However, value is an abstract term and can be interpreted from different perspectives, such as economic or social, by different stakeholders, such as the social media companies or the social media users \citep{zhu2018value, umbrello2021mapping, cummings2006integrating}.
%
In this project, we mainly focus on integrating human-centred social values from the perspective of social media users since they are the most affected by the consequences of hate speech.
%
We also focus on hate speech detection as a binary classification problem, where a prediction for a social media post is either positive (hateful) or negative (neutral).
%

\section{Research questions}
%
The idea of most ML models with a reject option is that we reject predictions when the model's confidence is too low.
%
First, we need a metric that measures the total value of an ML model with a reject option based on the context-dependent values.
%
By optimizing the value of this metric, we can retrieve an optimal confidence threshold that we can use to determine when we can accept ML predictions or if we need to reject them.
%
Second, we need to find out how we can retrieve context-dependent values for the task of hate speech detection.
%
The goal is to retrieve the value ratios by which we mean, for example, the ratio between an FP and an FN prediction.
%
Therefore, our research questions are as follows:
%

\begin{flushleft}
	\textbf{RQ} How can we reject predictions of machine learning models in a value-sensitive manner for hate speech detection?
	\begin{itemize}
		\item \textbf{SRQ1} How can we measure the total value of machine learning models with a reject option?
		\item \textbf{SRQ2} How can we determine the value ratios between rejections and true positive (TP), true negative (TN), false positive (FP), and false negative (FN) predictions?
	\end{itemize}
\end{flushleft}

\section{Our work}
This thesis research tackled the problems of proactive moderation by creating a human-AI solution where the advantages of humans (cognitive abilities and ability to make judgements) and machines (automation and performance) are combined \citep{woo2020future}.
%
We did this by proposing a \emph{value-sensitive} rejector for detecting hate speech that rejects ML predictions when the prediction's confidence is too low based on human-centred values.
%
To the best of our knowledge, ML with rejection has not been used in hate speech detection before.
%

%
In this thesis, we conducted a survey study where we recruited 160 participants to evaluate their perception of 40 different hate speech detection scenarios.
%
Each scenario simulates either a TP, TN, FP, FN, or rejected prediction in the context of hate speech detection.
%
We carefully selected 40 social media posts through an extensive content analysis procedure for creating the scenarios.
%
We proposed using the magnitude estimation (ME) rating scale for retrieving the value ratios by measuring the user perception of all scenarios.
%
We validated the ME scale by conducting a separate survey study with a bounded scale comprising 100 rating levels, called the 100-level scale.
%
The results show a high inter-rater agreement between the participants of the ME survey, indicating that ME is suitable for retrieving human-centred values.
%
We also found that users tend to agree more with correct predictions than the degree of disagreement with incorrect predictions, implying that social media users highly appreciate correct predictions made by the social media platform.
%
Additionally, we found that participants agree more with each other for incorrect predictions than for correct predictions, indicating a strong consensus over the harm.
%
Finally, analysis of the demographical features showed that for most scenarios, there are no differences in the user perception between different demographic groups.
%

%
We created a value-sensitive metric that measures the total value of an ML model for some confidence threshold based on the value ratios and a set of predictions.
%
We can convert any ML model into a value-sensitive rejector by finding the optimal confidence rejection threshold for which the value-sensitive metric achieves the maximum value.
%
We applied the value-sensitive rejector, with the value ratios from the survey study as its input, to two different datasets and three state-of-the-art hate speech classification models: a traditional, a deep learning, and a BERT-based model \citep{devlin2018bert}.
%
We denote the first dataset as the \emph{seen} dataset, a test dataset from the same source as the training dataset of the models.
%
We denote the second as the \emph{unseen} dataset, a test dataset from a different source, to simulate how the models would perform in the real world on new data.
%
The results demonstrate that the value-sensitive rejector can be beneficial for hate speech detection since we show that we maximize the values of several hate speech detection models by rejecting predictions.
%
The results with the \emph{unseen} data show that hate speech detection models are susceptible to bias, which confirms the findings from related studies.
%
Finally, the results show that when selecting the optimal model, using the value of our value-sensitive metric as the optimization target might return different results than using accuracy.
%

\section{Contributions}
In summary, we make the following contributions:
%
\begin{itemize}
	\item We introduce the concept of rejecting machine learning predictions into the task of hate speech detection;
	\item We introduce the magnitude estimation scale for measuring user perception to correct and incorrect machine learning predictions;
	\item We present a value-sensitive metric for measuring the total value of a machine learning model with a reject option;
	\item We demonstrate through a survey study that the magnitude estimation scale is suitable for retrieving the value ratios of TP, TN, FP, FN, and rejected predictions in the context of hate speech detection;
	\item We demonstrate that our value-sensitive rejector can guide us in determining when to accept or reject machine learning predictions to obtain optimal model values;
\end{itemize}

\section{Thesis outline}
%
In this thesis report, we first discuss the related work in chapter \ref{ch:related-work}.
%
Then in chapter \ref{ch:rejector}, we present the design of the value-sensitive rejector.
%
Chapter \ref{ch:survey} explains the design of the survey study.
%
In chapter \ref{ch:results}, we present the results of the experiments of the survey study and the value-sensitive rejector.
%
Finally, chapter \ref{ch:discussion} discusses the results, and chapter \ref{ch:conclusion} contains the conclusion.
