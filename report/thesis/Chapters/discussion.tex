\chapter{Discussion}
\todo[inline]{Answer research questions}

\todo[inline]{Discuss future work}

\section{Survey study}

\section{Value-sensitive rejection}

\section{Implications}
\todo[inline]{Explain that \cite{olteanu2017limits} claims that we need more human-centred metrics instead of abstract metrics such as precision and we agree with that by introducing our own human-centred metric}

\section{Limitations}
\todo[inline]{Hate speech is difficult domain as there tend to be a lot of disagreement between people about what is considered hate speech and what not. \citet{ross2017measuring} found low Krippendorff alpha values in a hate speech survey. So our findings are in line with theirs.}
\todo[inline]{Explain limitations of the metric and the survey study}
\todo[inline]{The rejection threshold is calculated using the test set. This test set needs to be as realistic as possible. Furthermore we need to have calibrated models since we rely purely on the confidence values. This is also hard to realize. Temperature scaling can help, but it is still limited.}

\section{Recommendations}
\todo[inline]{Magnitude Estimation seems promising for future research in HCI.}
\todo[inline]{Personal and demographic characterisitcs might have  a big impact. So further analysis on those aspects seem relevant.}
\todo[inline]{Perhaps we can train ML models using the values of TP, TN, FP, FN, rejection in an integrated rejector. So we train the ML model and the rejector simultanously using the values from the survey. So then during training, the FN predictions are punished more than FP predictions.}